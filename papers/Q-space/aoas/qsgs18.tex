\documentclass[dvips,aoas,preprint]{imsart}

\RequirePackage[OT1]{fontenc}
\RequirePackage{amsthm,amsmath,natbib}
\RequirePackage[colorlinks,citecolor=blue,urlcolor=blue]{hyperref}
%\RequirePackage{hypernat}
\usepackage{amssymb,bm,graphicx}

% settings
%\pubyear{2005}
%\volume{0}
%\issue{0}
%\firstpage{1}
%\lastpage{8}
\arxiv{math.PR/0000000}

\startlocaldefs
\numberwithin{equation}{section}
\theoremstyle{plain}
\newtheorem{thm}{Theorem}[section]
\endlocaldefs

\newcommand{\var}{\ensuremath{\text{var}}}
\newcommand{\E}{\ensuremath{\text{E}}}
\newcommand{\be}{\mathbf{e}}
\newcommand{\bp}{\mathbf{p}}
\newcommand{\bq}{\mathbf{q}}
\newcommand{\bE}{\mathbf{E}}
\newcommand{\q}{\mathbf{q}}
\newcommand{\tbq}{\tilde{\bq}}
\newcommand{\Q}{\mathbf{Q}}
\newcommand{\br}{\mathbf{r}}
\newcommand{\bu}{\bs{\upsilon}}
\newcommand{\uu}{\mathbf{u}}
\newcommand{\x}{\mathbf{x}}
\newcommand{\X}{\mathbf{X}}
\newcommand{\y}{\mathbf{y}}
\newcommand{\cA}{\mathcal{A}}
\newcommand{\cF}{\mathcal{F}}
\newcommand{\cG}{\mathcal{G}}
\newcommand{\cH}{\mathcal{H}}
\newcommand{\cI}{\mathcal{I}}
\newcommand{\cK}{\mathcal{K}}
\newcommand{\cQ}{\mathcal{Q}}
\newcommand{\R}{\mathcal{R}}
\newcommand{\cS}{\mathcal{S}}
\newcommand{\vare}{\varepsilon}
\newcommand{\bld}[1]{\mathbf{#1}}
\newcommand{\bs}[1]{\boldsymbol{#1}}
\newcommand{\wh}[1]{\widehat{#1}}
\newcommand{\wt}[1]{\widetilde{#1}}
\newcommand{\ol}[1]{\overline{#1}}

\begin{document}

\begin{frontmatter}
\title{A Statistical Framework for High Angular Resolution Diffusion Imaging in {\em q}-Space}
\runtitle{HARDI Geometry and Statistics}

\begin{aug}
\author{\fnms{Sofia C.}\snm{Olhede}\thanksref{t1,m1}\ead[label=e1]{s.olhede@ucl.ac.uk}}
\and
\author{\fnms{Brandon}\snm{Whitcher}\thanksref{m2}\ead[label=e2]{brandon.j.whitcher@gsk.com}}

\thankstext{t1}{Sofia Olhede is with the Departments of Computer Science and Statistical Science.}
\runauthor{S. C. Olhede and B. Whitcher}

\affiliation{University College London\thanksmark{m1} and GlaxoSmithKline\thanksmark{m2}}

\address{Department of Statistical Science\\
University College London\\
Gower Street\\
London WC1 E6BT\\
United Kingdom\\
\printead{e1}}

\address{Clinical Imaging Centre\\
GlaxoSmithKline\\
Hammersmith Hospital\\
Imperial College London\\
Du Cane Road\\
London W12 0HS\\
United Kingdom\\
\printead{e2}}
\end{aug}

\begin{abstract}
High angular resolution diffusion imaging data is the observed
characteristic function for the local diffusion of water molecules in
tissue.  These data are used to infer the organization of white matter
fiber bundles in structural brain imaging.  Non-parametric scalar
measures are proposed to summarize such data, and to locally
characterize spatial features of the diffusion probability density
function (PDF), relying on the geometry of the characteristic
function.  Summary statistics are defined so that their distributions
are, to first order, both independent of nuisance parameters and
analytically tractable.  The dominant direction of the diffusion at a
spatial location (voxel) is determined, and a new set of axes are
introduced in Fourier space.  Variation quantified in these axes
determines the local spatial properties of the diffusion density.
Non-parametric hypothesis tests for determining whether the diffusion
is unimodal, isotropic or multi-modal are proposed.  More subtle
characteristics of white-matter microstructure, such as the degree of
anisotropy of the diffusion~PDF and symmetry compared with a variety
of asymmetric PDF alternatives, may be ascertained directly in the
Fourier domain without parametric assumptions on the form of the
diffusion~PDF.  We simulate a set of diffusion processes and
characterize their local properties using the newly introduced
summaries.  We show how complex white-matter microstructure that spans
multiple voxels exhibits clear ellipsoidal and asymmetric structure in
simulation, and assess the performance of the statistics in
clinically-acquired MRI data.
\end{abstract}

%\begin{IEEEkeywords}
%Anisotropy, asymmetry, diffusion-weighted magnetic resonance imaging
%(DW-MRI), high angular resolution diffusion imaging (HARDI),
%non-parametric statistics.
%\end{IEEEkeywords}

\begin{keyword}
\kwd{anisotropy}
\kwd{asymmetry}
\kwd{magnetic resonance imaging}
\kwd{diffusion weighted imaging}
\kwd{non-parametric}
\end{keyword}

\end{frontmatter}

\section{Introduction}
\label{intro}

Many applications in brain imaging are based on calculating local
statistics that are later combined to infer global properties of
spatial links or connections. In this paper we focus on the local
analysis of high angular resolution diffusion imaging (HARDI) data, a
special type of magnetic resonance imaging (MRI).  HARDI observations
correspond to the local (voxel-wise) measurement of the Fourier
transform of the local water molecule diffusion probability density
function (PDF) \citep{Callaghan} in a number of different orientations
over a spherical shell of fixed radius.

A HARDI acquisition scheme permits the characterization of directional
spatial properties of the diffusion PDF.  The local structure of brain
tissue can be inferred from such measurements
\cite{basser1994,bas:relationships}.  Once local statistics have been
formed, it is of interest to combine information across voxels
(spatial locations), to track nerve fiber bundles (e.g. connect local
directions of estimated diffusion PDFs to recognize nerve connections)
or to infer local fiber structure from the estimated diffusions
\citep{mor-zij:fiber-tracking}, and/or to use other locally-defined
summaries in inference procedures \citep{jen-etal:kurtosis}.

Different orientational sampling designs can be used at each location
(or voxel) and if a simple parametric model is used for the
diffusion~PDF, then rather sparse sampling will be sufficient to
recover the parameters of the model.  Traditional analysis of HARDI
measurements is based on modelling the diffusion~PDF parametrically as
a (zero-mean) Gaussian, and estimating a diffusion tensor (the
covariance matrix of the Gaussian PDF), a procedure which corresponds
to diffusion tensor imaging (DTI) \citep{basser1994}.  Such methods
have drawbacks, for example not describing more complex fiber
structures very well, and their usage trades a small variance for
potentially large biases.  While the diffusion tensor model has both
theoretical justification and has been extremely popular, it prohibits
one from describing more complicated (white-matter voxel) structure,
such as crossing, kissing and forking fibers
\citep{mor-zij:fiber-tracking}.

It is believed that intravoxel orientational heterogeneity affects as
many as one third of all imaged white-matter voxels
\citep{Behrens2007}, and so addressing such structure is important.
With more time-intensive sampling schemes (such as HARDI
\citep{tuc-etal:high} or diffusion spectrum imaging) various more
complicated estimators can be used, e.g. multi-tensor modelling
\citep{Alexander2005}, and non-parametric alternatives such as
persistent angular structure MRI \citep{Jansons}, Q-ball imaging
\citep{Tuch}, the diffusion orientation transform
\citep{Ozarslan2006}, and spherical deconvolution \citep{Tournier}.
% with various proposed improvements \citep{Hess,Descoteaux,Kaden2007,Jian2007}
While a non-parametric approach may remove bias, usage of such
non-parametric methods is challenging because the diffusion is
measured in the Fourier domain ($q$-space\footnote{$q$-space is the
  Fourier domain representation of the local diffusion and is the
  space where measurements are made.  The global image Fourier
  representation is usually inverted to a spatial representation, but
  the local Fourier transform is not inverted as part of the
  acquisition, leaving the spatial domain observations associated with
  a measurement of local diffusion in a Fourier domain orientation.}),
and the characteristic function has been considerably undersampled for
scanning times that may be used in the clinic.  This challenges the
stable inversion of information (the local characteristic function) to
local spatial structure.

This paper develops a statistical framework for characterising HARDI
data directly in $q$-space \citep{tuc-etal:high} without local
inversion.
%  We focus on describing spatial structure
%of the local diffusion probability density function (PDF) directly in
%$q$-space and define estimators of
%summaries of spatial structure. 
This avoids calculating non-linear transformations of the data, whose
usage usually leads to intractability of the distributions of
calculated data summaries.  The approximate distributions of the
proposed estimators in this paper are derived and these are defined so
that to first order they are free of any nuisance parameters.  The
proposed statistics are a first step towards automated detection of
subtle characteristics of white-matter microstructure that go beyond
the typical measurements; i.e., scalene diffusions or asymmetry in
decay in a fixed axes.  Both properties, scalene diffusion and
asymmetry, have been found in a forking fiber structure, and may be
important summaries to feed into fiber-tracking algorithms
\citep{mor-zij:fiber-tracking}.  The derived methods also serve as a
warning when interpreting multi-tensor models in clinically-feasible
acquisition schemes, as similar characteristics can be obtained from
more complex single peaked structures.

% The revival of connectional anatomy, through the application of in
% vivo assessment of white-matter structure using Diffusion Tensor
% Imaging (DTI), quickly highlighted the limitations of the diffusion
% tensor as a model \citep{MarcoCatani02242007}.  When applying MRI as
% a biomarker to characterize a pathological process one must worry
% about various concepts like accuracy, precision and reproducibility.
% DTI has already been applied to a wide variety of neurological
% disorders; for example, multiple sclerosis, ischemic leukoaraiosis,
% CADASIL and Alzheimer's disease \citep{Horsfield2002}.  Psychiatric
% applications of DTI are also well documented \citep{Lim2002}.
% Taking the notion of an imaging biomarker further leads one to
% consider performing a clinical trial that introduces a
% disease-modifying treatment to subjects.  In this situation it is
% crucial to have a biologically-relevant biomarker that is extremely
% sensitive to changes due to the disease process or treatment
% \citep{whi-mat:pitfalls}. Better sensitivity translates directly
% into an increase in power to detect changes between groups of
% subjects in a clinical trial setting. Direct consequences of this
% are lower numbers of subjects that must be recruited or a reduction
% in the length of the trial -- either of these have the potential to
% speed up the drug development process.  The methodology presented
% here will provide researchers with the ability to quantify features
% from HARDI data that are currently unavailable from current
% techniques.  Specific links between white-matter features (such as
% crossing and bending fibers) and $q$-space characteristics will be
% provided.

%Multi-tensor
%modelling either requires the number of tensors to be known
%\citep{Seunarine}, or additional constraints are required for
%consistent estimation of the diffusion PDF to ensure identifiability.
%In order for such methods to work well the Gaussian mixture model must
%adequately represent the observed structure of the data.
%
%PAS-MRI and Q-ball, competitive non-parametric methods, are fitted and
%smoothed using an implicit smoothness parameter.  Q-ball methods
%usually assume a greater number of observations than are normally
%collected in a clinical acquisition protocol, whilst PAS-MRI seeks to
%determine the orientation associated with a fixed spatial scale.
%Furthermore, usually isotropic smoothing is used when implementing both %these
%methods which may eliminate interesting and important high-frequency
%structure present in the underlying diffusion PDF; i.e., microscale
%features.  Finally, the DOT is calculated under the assumption of
%monoexponential decay, and the radial dependence is integrated out

%analytically, using the Stejskal-Tanner relationship.  This makes the
%spatial interpretation highly model dependent and therefore
%susceptible to model specification error.  The integral is
%subsequently expanded into a Laplace series from which the average
%displacement probability can be deduced.  Expansions of this sort are
%necessarily truncated for implementation which corresponds to
%smoothing with an implicit bandwidth. 

%Identifying multimodal distributions of diffusion can, to a great
%extent, be addressed by a variety of methods
%\citep{Jansons,Tuch,par-ale:pico,hos-wil-ans:inference,beh-etal:multiple}.

Global features like bi- or multi-modality of the diffusion~PDF are
described reasonably well by many methods over a range of
signal-to-noise values, with the small caveat that the various
implicit assumptions inherent to any of the given methods must be
satisfied.  Parametric models introduce bias when they are not
appropriate, whereas using a nonparametric method increases the
variance in the estimation.  Using a moderate number of directions in
the HARDI sampling scheme restricts the possibility of determining
smaller scale structure of the diffusion~PDF.  Strong parametric
assumptions increase the power of any proposed statistic to detect
multiple diffusion directions, with the consequence that any deviation
from the prescribed structure in the parametric model may be used to
reject null hypotheses such as unimodality.
% This implies that incorrect conclusions may be reached when
% interpreting the properties of the PDF from such parametric models.

%The first task at a given voxel is to determine if the diffusion can
%be described as strongly directional versus isotropic or multimodal
%and we develop appropriate non-parametric procedures for this task.
%Alexander {\em et al.} \citet{ale-bar-arr:detection} developing related
%methods modelled the hypothesis of isotropy versus a single peak in
%terms of spherical harmonics, where the hypotheses were defined by the
%order of the decomposition.  With a fixed sampling scheme, the ability
%to detect a peak depends on the directional alignment of the great
%circle in $q$-space that is associated with the peak and the sampling
%points in $q$-space.  This fact is related to the occasional poor
%performance of peak detection methods when the sampling grid is
%misaligned with the true peaks, as noted by \citet{Jian2007}.

In the method proposed here to determine the properties of the
diffusion~PDF, the directional PDFs (i.e., diffusion~PDFs that are
strongly elongated in a given direction) are separated from the
isotropic PDFs (non-preferential directional structure), using a test
based on a comparison of relative magnitudes in $q$-space.
Subsequently, multimodal distributions are sifted from the isotropic
and unidirectional.  The unidirectional diffusion is associated with a
great circle in $q$-space \citep{Tuch}, and we call this the {\em
  dominant great circle}.  The strongest direction defines an
important spatial summary of the diffusion~PDF, and specifies the
major axis of the diffusion in $q$-space.  The perpendicular to the
major direction in space defines a set of points lying on a great
circle in $q$-space, which exactly corresponds to the dominant great
circle.

If a given voxel has been diagnosed as unidirectional (or if there is
a dominant great circle in $q$-space) then we seek to characterize its
main unidirectional structure in more detail.  A scalar measure
resembling the popular fractional anisotropy\footnote{The fractional
  anisotropy (FA) is a measure of uniformity of the eigenvalues of a
  Gaussian covariance matrix \citep{bas:FA}.}  statistic is defined as
the {\em anisotropy statistic}, by comparing the magnitude of the
$q$-space diffusion on the dominant great circle with its two
perpendicular point(s).  This measure determines the degree of
anisotropy of the diffusion~PDF.  Further investigation of
unidirectional voxels causes us to focus on quantifying the uniformity
of decay in the minor axes of the diffusion~PDF, or the perpendicular
to the dominant great circle, to describe further detailed structure
of the characteristic function.

Ellipsoidal diffusions are an important class of diffusion processes
and the scalene structure of the diffusion~PDF is particularly
important when combining voxel-wise information \citep{Seunarine}.
The aforementioned work showed that the scalene structure of the peak
is related to the peak anisotropy in space and important for treating
bending and fanning fibers.  \citet{Seunarine} used the peak Hessian
to improve PICo \citep{par-ale:pico}, a fiber tracking method, but
such information can feed into any tracking algorithm of choice.  For
diffusions with ellipsoidal decay, their minor axes are well defined
by this (scalene) decay structure, while for non-ellipsoidal
diffusions the minor axes correspond to a set of axes in the plane of
the dominant great circle, parameterising locations on the dominant
great circle.  We examine the scalene structure of the PDF, which is
quantified by the difference in decay in the two spatial minor axes,
defined as such also for non-ellipsoid diffusions.
% This corresponds to examining the variability of the diffusion on
% the great circle perpendicular to the vector associated with the
% major direction of the diffusion.  For a Gaussian diffusion this is
% given by the two minor eigenvalues of the eigen-decomposition of the
% diffusion tensor.
A statistical test for uniformity on the great circle is developed
that can be related to the spatial decay of the diffusion~PDF in the
minor axes.  Another feature of interest in the PDF is asymmetry in
the decay in a fixed direction perpendicular to the dominant great
circle.  This heuristic may be visualized in space as a diffusion~PDF
that appears ellipsoidal but the peak is in one of the foci rather
than the center of the ellipse.  We introduce a test statistic for
asymmetry based on this understanding.

% To motivate our interest in asymmetry and ellipsoidality we simulate
% forking and crossing structures, and show how both asymmetry and
% ellipsoidality follows as precursors to forking structure, and such
% information could be used to improve the tracking of fibers.

The methodology presented here improves understanding of the
diffusion~PDF by not relying on parametric assumptions when analysing
the measurements, yet still relating $q$-space structure directly to
spatial properties.  Non-parametric statistical summaries are defined
directly in $q$-space to increase power of the proposed hypothesis
tests and theoretical critical values for the statistics are provided.
Understanding the inherent limitations of HARDI measurements can be
obtained directly from our discussion of simulated diffusions, thus
increasing our understanding of the parametric assumptions that are
necessary to derive more complicated structures from the
diffusion~PDF.

\section{Statistical Models for HARDI Data}

\subsection{Observational Model}

We denote the sampling of the observations by the set
$\cQ_0=\left\{\tbq_i\right\}_{i=1}^n$.  At each
$\tbq_i=(\tilde{q}_{i1},\tilde{q}_{i2},\tilde{q}_{i3})$ on the unit
sphere
$\|\tbq\|=(\tilde{q}_{1}^2+\tilde{q}_{2}^2+\tilde{q}_{3}^2)^{1/2}=1$
we obtain an observed measurement $\wt{A}(\tbq_i)\ge0$, corresponding
to the magnitude of a complex-valued observation (the noisy characteristic function of local diffusion\footnote{Note that this is different from the empirical characteristic function.}).  Furthermore we take
$n_0$ observations at $\q=\bld{0}$, denoted by $\wt{A}_k(\bld{0})$ for
$k=1,\dots,n_0$.  We distinguish here between the measured apparent
diffusion at $\tbq_i$, namely $\wt{A}(\tbq_i)$, and the theoretical
diffusion value, $\cA(\tbq_i)$. Note that the expected value of
$\wt{A}(\tbq_i)$ is {\em not} equivalent to $\cA(\tbq_i)$, for two
reasons. Firstly because the observations are magnitudes, with the noise
contributing in the expectation, and secondly we need to renormalize
the observed diffusion to have unit volume, as noted by \citet{Alexander2005}. As the PDF
is a density it has to satisfy the normalization of
\begin{equation}
  \int \int \int a(\x) \, d^3\x = 1 ~ \Rightarrow ~ \cA(\bld{0}) = 1,
\end{equation}
where $a(\x)$ is the diffusion probability density function (PDF).  We
apply a biased estimator of a simple average to estimate the inverse
of the normalizing constant by
$\ol{A}(\bld{0})=\sum_{k=1}^{n_0}\wt{A}_k(\bld{0})/n_0$.  We
renormalize the observed diffusion such that
$A(\tbq_i)=\wt{A}(\tbq_i)/\ol{A}(0)$.  $A(\tbq_i)$ has (approximately)
a Rician distribution with parameters $\cA(\tbq_i)$ and $\sigma^2$ \citep{Gudbjartsson}.
As the SNR will be large at $\q=\bld{0}$, the noise floor of the
Rician distribution will have limited impact in the estimation of the
normalization.

The normalized diffusion measurements $\cA(\tbq_i)$ should exhibit
symmetry as the PDF is real-valued, symmetric and indeed positive;
i.e., $\cA(-\tbq_i)=\cA(\tbq_i)$ (see \citep{Wedeen05}).  To fully exploit the Hermitian
symmetry we shall reflect the observations to the augmented set
$\cQ=\{\q:\q\in\cQ_0\}\cup\{\q:-\q\in\cQ_0\}$, and set
$A(-\tbq_i)=A(\tbq_i)$ as was also done by \citet{Jansons}.  

We assume that a non-parametric estimator of the diffusion in
$q$-space is constructed. We chose to use a variable bandwidth estimator, see
\citet{OlhedeWhitcher,OlhedeWhitcherISBI}, but the outlined methodology
is applicable to other linear estimators, and can be extended to for
example radial basis functions and/or spherical harmonics, if with
some straightforward alteration of the statistical properties (second order structure) of the
estimator.

\subsection{Great Circles in $q$-Space}

Spatial properties of the diffusion PDF may be described directly in
$q$-space.  The advantage of such an operation is that we avoid the
need to invert the PDF to the spatial domain for analysis, allowing us
to employ a broad range of modelling approaches.  A basic building
block of our analysis is an {\em ellipsoid density}.  We refer to a
density $a_E(\x)$ as an ellipsoid density if its FT takes the form
\begin{equation}\label{ellipsoid}
  \cA_E\left(\q; \bm{\Lambda},\bm{\Upsilon}\right) =
  B\left(\sqrt{\sum_{j=1}^3\lambda_j\left|\bu_j^T\q\right|^2
  }\right),
\end{equation}
where $\lambda_j\ge{0}$, $j=1,2,3$, $\{\bu_j\}$ constitutes a basis
for ${\mathbb{R}}^3$ and $B(\cdot)$ is a monotonically decreasing
function.  For example, commonly used is the Gaussian characteristic function, with
$B(q)=e^{-2(\pi q)^2}$.  We collect the eigenvalues in the matrix
$\bs{\Lambda}={\text{diag}}\left(\lambda_1,\;\lambda_2,\;\lambda_3\right)$,
and define
\[\bm{\Upsilon}^T=\{[\upsilon_{11}\;\upsilon_{12}\;\upsilon_{13}];
[\upsilon_{21}\;\upsilon_{22}\;\upsilon_{23}];
[\upsilon_{31}\;\upsilon_{32}\;\upsilon_{33}]\},\] to model the axis of any orientational structure.  Ellipsoid densities
are natural building blocks, just like the special case of the DTI
model, but do not (for example) include multimodal densities.  If the $q$-space
density takes this form, then the spatial PDF is given by inverting
the FT
\begin{equation}
  a_E(\x;\bm{\Lambda},\bm{\Upsilon}) = \int\int\int_{\mathbb{R}^3}
  \cA_E(\q; \bm{\Lambda},\bm{\Upsilon})e^{i2\pi \q^T\x} \, d^3 \q
\end{equation}
\citep{Callaghan}.  We note for $\x\in\mathbb{R}^3$, with $x=\|\x\|$
and $q=\|\q\|$, that $a_E(\x;\bm{\Lambda},\bm{\Upsilon})$ takes the
form
\begin{eqnarray}\label{ellipsoid2}
  a_E(\x;\bs{\Lambda},\bs{\Upsilon}) &=& |\bs{\Lambda}|^{1/2} \,
  b\left(\left\|\bs{\Lambda}^{-1/2}\bs{\Upsilon}\x\right\|\right),\\
  b(x) &=& \int_{-\infty}^{\infty} \int_{-\infty}^{\infty}
  \int_{-\infty}^{\infty} B(q) e^{i2\pi\q^T\x} \, d^3\q\\
  &=& \frac{1}{2\pi^2 x}\int_0^{\infty}
  B\left(\frac{q'}{2\pi}\right)\sin(xq')\,q'\,dq'\\
  &=& \frac{2}{x}\int_0^{\infty}B(q)\sin(2\pi xq) \,q\,dq,
  \nonumber
\end{eqnarray}
which follows from \citep{grad}[p.~1112].  The meaning of ``ellipsoid
density'' becomes clear from this expression, since whenever
$\|\bs{\Lambda}^{-1/2}\bs{\Upsilon}\x\|=R$, where $R\ge 0$ is a
constant, the function $a_E(\x)$ takes the same value in space.  As
long as all the eigenvalues are positive, $ a_E(\cdot)$ will therefore
map out ellipsoid contours of equal function value in space.  The Gaussian DTI model fits into this class of
densities with $b(x)=e^{-x^2/2}/(2\pi)^{3/2}$ as do, for example, the
Mat\'ern family with the spatial variable exchanged with the spatial
frequency variable \citep{Matern}.  The model proposed by
\citet{Kaden2007} is also related to such densities. Typical shapes are
found in Figure~\ref{fig:inrha} where we show a prolate ellipsoid
distribution, and its Fourier transform ((a) and (e)), a scalene
ellipsoid PDF ((b) and (f)), a mixture of ellipsoids ((c) and (g)),
and a non-ellipsoid ((d) and (h)). The values of $\bm{\Upsilon}$
specify the orientation of the PDF, while $\bm{\Lambda}$ gives its
qualitative appearance, when coupled with $B(\cdot)$. Unfortunately,
looking directly at Figure \ref{fig:inrha} we do not get the clearest
feeling for the local structure near the peak, and so additional ways of
characterizing the PDF will be formulated.

%%%%%%%%%%%%%%%%%%%%%%%%%%%%%%%%%%%%%%%%%%%%%%%%%%%%%%%%%%%%%%%%%%%%%%%%%%
\begin{figure}[!htbp]
  \begin{minipage}[]{0.10\textwidth}
    \centering
    (a)
    \includegraphics*[width=\textwidth]{inspace1.ps}
  \end{minipage}
  \begin{minipage}[]{0.10\textwidth}
    \centering
    (b)
    \includegraphics*[width=\textwidth]{inspace2.ps}
  \end{minipage}
  \begin{minipage}[]{0.10\textwidth}
    \centering
    (c)
    \includegraphics*[width=\textwidth]{inspace3.ps}
  \end{minipage}
  \begin{minipage}[]{0.10\textwidth}
    \centering
    (d)
    \includegraphics*[width=\textwidth]{inspace0.ps}
  \end{minipage}
  \begin{minipage}[]{0.10\textwidth}
    \centering
    (e)
    \includegraphics*[width=\textwidth]{infreq1.ps}
  \end{minipage}
  \begin{minipage}[]{0.10\textwidth}
    \centering
    (f)
    \includegraphics*[width=\textwidth]{infreq2.ps}
  \end{minipage}
  \begin{minipage}[]{0.10\textwidth}
    \centering
    (g)
    \includegraphics*[width=\textwidth]{infreq3.ps}
  \end{minipage}
  \begin{minipage}[]{0.10\textwidth}
    \centering
    (h)
    \includegraphics*[width=\textwidth]{infreq0.ps}
  \end{minipage}
  \caption[]{Unimodal densities ((a), (b) and (d)) in space and ((e),
    (f) and (h)) frequency, as well as a mixture of ellipsoids (c) in
    space, and (g) in frequency.}
  \label{fig:inrha}
\end{figure}
%%%%%%%%%%%%%%%%%%%%%%%%%%%%%%%%%%%%%%%%%%%%%%%%%%%%%%%%%%%%%%%%%%%%%%%%%%

\subsection{The Orientation Distribution Function}

An important tool in understanding HARDI data is via the Orientational
Distribution Function (ODF).  The ODF quantifies the directional
structure of the diffusion PDF {\em in space}.  A popular object of
study, it corresponds to several {\em different} functions.
\citet{Tuch} and \citet{Hess,Descoteaux} define the ODF to be
\begin{equation}
  \text{ODF}_T(\theta,\phi) = \frac{1}{Z} \int_0^{\infty} a(r\uu)\,dr,
\end{equation}
where $\x=r\uu$, $\|\uu\|=1$ and $Z$ is a normalizing constant.  A
non-linear transformation is necessary for the ODF to have a more
peaked and clear directional structure.  Because this is not a true
marginalization of a PDF (the increment needs a weighting by $r^2$), and weighs very low scales heavily, the
diffuse directional structure of the large scale structure smooths the
marginal PDF of orientations, giving it a ``blunted'' appearance.
\citet{Wedeen05} define the ODF as the truly marginalized PDF over all
spatial radii 
\[\text{ODF}_W(\theta,\phi)=\int_0^{\infty}r^2a(r\uu)\,dr.\]  An
alternative version may be found in \citep{Jansons}, where the
orientational structure associated with a single radius is fitted to
the observed data (i.e. the Persistent Angular Structure (PAS-MRI)
algorithm).  It is useful to note that the observed data is not
associated purely with a single radius, and for this to be a
mathematically correct procedure the observed HARDI measurements
should be convolved with a suitable kernel prior to estimation.
Despite this fact PAS-MRI usually gives good results in practice.  All three
of these orientational summaries are measuring {\em different} properties of
the directional structure of the data, and only
$\text{ODF}_W(\cdot,\cdot)$ is a true marginal PDF.
% Only the Wedeen ODF is a true marginal PDF, and we doubt a true
% interest in structure across all radii -- surely information below a
% lower band is purely attributed to noise, and really high scale
% orientational information blurs out the structure we want.  
%The modelling simplifications of each method are a necessity because of %undersampling, and in
%each case are appropriately chosen.
% as we have undersampled the data and cannot with a simple set of
% HARDI measurements perfectly calculate the ODF without simplifying
% assumptions or regularization. 

Another directional representation of diffusion data corresponds to
the spherical convolution model \citep{Tournier}.  In this model,
$q$-space observations are modelled as convolved fiber ODFs, and fiber
populations are estimated using deconvolution methods. The magnitudes
are then not comparable with previously defined estimators of ODFs.
Extensions to these methods have been proposed: by modelling the ODF
as a mixture of Bingham distributions \citep{Kaden2007}, and by
regularizing the deconvolution problem by proposing the use of
constrained optimization methods \citep{Jian2007}.  The solution in
\citet{Kaden2007} is parametric and the theoretical assumptions
necessary to apply the regularized methods are (in general) violated
\citep{Jian2007}.

The ellipsoid PDF model in \eqref{ellipsoid2} can be extended into a
larger class of arbitrarily peaked deformed PDFs by taking
$\bm{\Lambda}(\x)={\text{diag}}(\lambda_{11}(\x),\lambda_{22}(\x),\lambda_{33}(\x))$,
$\lambda_{jj}(\x)\ge0$ for all $\x$, with $C$ a normalizing constant,
\begin{eqnarray}%\label{deformedellipse}
  a_{DE}(\x) &=& C \sqrt{|\bs{\Lambda}(\bs{\Upsilon}\x)|} \,
  b\left(\left\|\bs{\Lambda}(\bs{\Upsilon}\x)^{-1/2}
  \bs{\Upsilon}\x\right\|\right),\\ 
  a_{DE}(\bs{\Upsilon}^T\x) &=& C \sqrt{|\bs{\Lambda}(\x)|}
  b\left(\left\|\bs{\Lambda}(\x)^{-1/2}\x\right\|\right).
\end{eqnarray}
Because $\bs{\Lambda}(\x)$ is a diagonal matrix
$a_{DE}(\bs{\Upsilon}^T\x)$ exhibits the axes $(1,\;0,\;0)$,
$(0,\;1,\;0)$ and $(0,\;0,\;1)$.  Implementing a Fourier transform
directly with the change of variables, the Fourier transform is mixed
over the strengths in $\bs{\Lambda}(\x)$, but exhibits the same
orientational axes, if the ordering in magnitude of the eigenvalues does not switch over $\x$.  We have the model of
\begin{equation}\label{deformedell}
  {\cA}_{DE}(\q) = C \int\int\int_{\mathbb{R}^3}
  |\bs{\Lambda}(\x)|^{1/2} \,
  b\left(\|\bs{\Lambda}(\x)^{-1/2}\x\|\right)
  e^{-i2\pi(\bs{\Upsilon}\q)^T\x} \, d^3\x.
\end{equation}
This function can take the appearance of a deformed ellipsoid in
space, and may then exhibit a different pattern of decay to the left
and right of the dominant great circle in $q$-space.  For the regular
ellipsoid distribution $a_E(\x)$ if one eigenvalue is larger than the
two others (say $\lambda_1>\lambda_2\ge\lambda_3$) then the ellipsoid
density (or equally in the case of the deformed density if
$\inf_{\x}\lambda_1(\x)>\sup_{\x}\lambda_2(\x)$) will observe a
maximum at the values (compare with \eqref{ellipsoid})
{\tiny\begin{eqnarray}\label{qbeta}
  \q(\beta) = \left\{
  \begin{array}{lcr}
    \beta \bu_2+\sqrt{1-\beta^2}\,\bu_3 & \text{if} & \beta\in[-1,1]\\
    \text{sgn} \left(\beta\right) (|\beta|-1)\bu_2 -
    \sqrt{1-(|\beta|-1)^2}\,  \bu_3 & \text{if} &
    \beta\in[-2,2]\backslash[-1,1]
  \end{array} \right. ,
\end{eqnarray}}
noting the match to Figure \ref{fig:inrha}(e), (f) and (h).  This
maximum great circle in $q$-space corresponds to the perpendicular
vector $\pm\bu_1$ in space, where the diffusion PDF exhibits a
maximum.  The structure near the peak ($\x=\pm\bu_1$) is mapped to a
structure contiguous to the great-circle; i.e., $\q\approx\q(\beta)$,
and now compare Figure \ref{fig:inrha}(a) with (b) and (d), with the
corresponding frequency descriptions of \ref{fig:inrha}(e) with (f)
and (h).  For this reason it is desirable to investigate the structure
of the PDF near the great circle of points $\{\q(\beta)\}$, in terms
of distances from the great circle to characterize structure in the decay from the main peak.  To obtain consistency in
notation, we define the set of points (or the great circle
perpendicular to $\bu$) via $\cG(\bu)=\{\q:\bu^T\q=0,\|\q\|=q\}$ and
$\cG(\bu_1)\equiv\{\q(\beta)\}$.  It is convenient to keep both sets
of notation for ease of exposition.

\section{Scalar Summaries and Test Statistics}

\subsection{Axes of Symmetry}

Before we can define appropriate scalar summaries in $q$-space,
additional axes to the $\beta$ axis \eqref{qbeta} are required.  For
any fixed vector $\q(\beta)\in\cG(\bu_1)$ we traverse a great circle
using the vectors
\begin{equation}\label{alphabetaplane}
  \q_\perp(\alpha,\beta) = \alpha\bu_1\pm\sqrt{1-\alpha^2}\q(\beta),
  \quad \alpha \in [-1,1],
\end{equation}
where for $\alpha\in\left[-2,2\right]\backslash[-1,1]$, the
corresponding expression may be formed as in \eqref{qbeta}.  Such a
great circle for a fixed value of $\beta$ will be referred to as a
{\em perpendicular great circle}.

An important component in the definition of our non-parametric
summaries is the {\em dominant great circle} $\cG(\x_\text{max})$ with
$\x_\text{max}$ given by
\begin{equation}
  \x_\text{max}=\arg \max_{\bu}\left\{ \oint_{\q\in\cG(\bu)}
  {\cA}\left(\q\right)\;d\q \right\}.
\end{equation}
If ${\cA}\left(\q\right)$ is an isotropic diffusion process, then
$\x_\text{max}$ is any vector in ${\mathbb{R}}^3$ with a fixed norm.
Alternatively, if ${\cA}\left(\q\right)$ is ellipsoid with
$\lambda_1>\lambda_2\ge \lambda_3$ then $\x_\text{max}=\bu_1$.  If
there are two fibers, then with relative weights of $a_1$ and $a_2$ of
fiber populations with $\bs{\Lambda}^{(1)}$ and $\bs{\Lambda}^{(2)}$
the individual eigenvalues
{\tiny \begin{equation}
  \x_\text{max} = \arg \max_{\bu}\left\{\left[ a_1
    \oint_{\q\in\cG(\bu)}
    {\cA}_E\left(\q;\bs{\Lambda}^{(1)},\bs{\Upsilon}^{(1)}\right) + a_2
    \oint_{\q\in\cG(\bu)}
    {\cA}_E\left(\q;\bs{\Lambda}^{(2)},\bs{\Upsilon}^{(2)}\right)\right]\;d\q
    \right\}.
\end{equation}}
For example if $a_1\gg{a_2}$ then $\x_\text{max}\approx\bu_1^{(1)}$,
or if $a_1=a_2=1/2$ and the great circles do not separate, then
$\x_\text{max}$ will lie precisely between the two maxima of the two
PDFs.  Once the great circles start to separate the maximum will go
with one of the two.

%%%%%%%%%%%%%%%%%%%%%%%%%%%%%%%%%%%%%%%%%%%%%%%%%%%%%%%%%%%%%%%%%%%%%%%%%%
\begin{figure}[!htbp]
    \begin{minipage}[]{0.32\textwidth}
      \centering
      (a)
      \includegraphics*[width=\textwidth]{pdfen11b.eps}
    \end{minipage}
    \begin{minipage}[]{0.35\textwidth}
      \centering
      (b)
      \includegraphics*[width=\textwidth]{pdfen11.eps}
    \end{minipage}\\
    \begin{minipage}[]{0.32\textwidth}
      \centering
      (c)
      \includegraphics*[width=\textwidth]{pdfen13b.eps}
    \end{minipage}
    \begin{minipage}[]{0.35\textwidth}
      \centering
      (d)
      \includegraphics*[width=\textwidth]{pdfen13.eps}
    \end{minipage}
  \caption{One- and two-dimensional summaries of Gaussian diffusion
    processes in $q$-space.  {\bf a,b:} Prolate diffusion process
    ($\lambda_1\gg\lambda_2=\lambda_3$). {\bf c,d:} Mixture of two
    Gaussian diffusion processes.  The dominant great circle is the
    solid line in the one-dimensional summaries {\bf a,c}, while the
    dotted line is the diffusion from a single perpendicular great
    circle for ({\bf a}) and the average perpendicular diffusion for
    ({\bf c}).  In the two-dimensional summaries ({\bf b} and {\bf d})
    all great circles perpendicular to the dominant great circle are
    plotted on the $y$-axis to form the $(\alpha,\beta)$ plane.}
  \label{fig:Gaussian-PDF}
\end{figure}
%%%%%%%%%%%%%%%%%%%%%%%%%%%%%%%%%%%%%%%%%%%%%%%%%%%%%%%%%%%%%%%%%%%%%%%%%%

\subsection{Degree of Non-Uniformity}

We represent a unidirectional Gaussian diffusion by plotting the value
of $\cA(\q(\beta))$ (solid line) for $\beta\in[-2,2]$ in
Fig.~\ref{fig:Gaussian-PDF}a.  The magnitude on the dominant great
circle is constant over different values of $\beta$ since
$\lambda_2=\lambda_3$.  To illustrate the difference in variation
across the dominant and perpendicular great circles we also plot the
value of $\cA(\q_{\perp}(\alpha,\beta))$ as a function of $\alpha$ for
a fixed $\beta$ (dotted line).  This line perfectly overlaps
$\cA(\q(\beta))$ at two locations, as it collides with the dominant
great circle when it wraps around the sphere, and decays symmetrically
from $\q(\beta)$.

We define a new coordinate system $(\alpha,\beta)$, where we expect
consistent variability in $\alpha$ and $\beta$, using our
parameterization of great circles \eqref{alphabetaplane}.  We plot the
unidirectional Gaussian diffusion $\cA(\q_{\perp}(\alpha,\beta))$ for
all perpendicular great circles in the plane
(Fig.~\ref{fig:Gaussian-PDF}b).  This prolate diffusion exhibits
variation only in $\alpha$, which is variation perpendicular to the
dominant great circle.  For the prolate diffusion example we can
therefore reduce the variance by averaging across $\beta$ and by
considering the function strictly in terms of $\alpha$.  We also plot
one-dimensional great-circle summaries for a mixture of two Gaussian
diffusions in Fig.~\ref{fig:Gaussian-PDF}c, where the dominant great
circle exhibits a large dynamic range relative to the perpendicular
great circles.  In fact, one can determine the number of peaks of the
diffusion PDF by comparing the dynamic range of the diffusion between
the dominant and perpendicular great circles.  For a complete picture
we also represent the multi-modal diffusion in the $(\alpha,\beta)$
plane in Fig.~\ref{fig:Gaussian-PDF}d, where variation is appreciable
in both the $\alpha$ and $\beta$ axes.

To overcome the need to compare the variation along the dominant great
circle with all perpendicular great circles individually, we define
the {\em average perpendicular diffusion} via
\begin{equation}\label{e:averagecont}
  \cA_\perp(\alpha) = \frac{1}{4}\int_{-2}^2
  \cA\left(\q_\perp(\alpha,\beta)\right) \, d\beta.
\end{equation}
One may also define the average perpendicular diffusion over a half
circle by pre-specifying a fixed location on the dominant great circle
and integrating in a window size $\pm1$ around this location.  This
will prevent certain features being masked by the Hermitian symmetry
of the $q$-space measurements.  If $\cA(\q)$ satisfies
\eqref{ellipsoid}, then we have
\begin{equation}\label{e:averagecont2}
  \cA_\perp(\alpha) = \frac{1}{4}\int_{-2}^2
  B\left(\sqrt{\lambda_1\alpha^2 + \left(1-\alpha^2\right) \left[\lambda_2
  \|\q(\beta)^T \bu_2\|^2 + \lambda_3 \|\q(\beta)^T\bu_3\|^2
  \right]}\right) \, d\beta.
\end{equation}
Thus, we are averaging the density function over small circles
parallel to the dominant great circle and $\cA_\perp(\alpha)$ measures
the average diffusion at a given value of $\alpha$.  In the special
case of $\lambda_2=\lambda_3$, then
\begin{equation}\label{e:averagecont3}
  \cA_\perp(\alpha) = \frac{1}{4} \int_{-2}^2 B\left(\sqrt{\lambda_1
  \alpha^2 + \lambda_2 \left[1-\alpha^2\right]}\right) \, d\beta =
  B\left(\sqrt{\lambda_1 \alpha^2 + \lambda_2 \left[1-\alpha^2\right]}\right).
\end{equation}
The average perpendicular diffusion $\cA_\perp(\alpha)$ provides a
useful summary of variation perpendicular to the dominant great
circle.  We define a summary of the PDF via
\begin{equation}\label{e:tau}
 \tau = \left[\frac{\max_{\alpha}\{\cA_\perp(\alpha)\}}
   {\min_{\alpha}\{\cA_\perp(\alpha)\}}\right] \bigg/
   \left[\frac{\max_{\beta}\{\cA(\q_{\perp}(0,\beta))\}}
     {\min_{\beta}\{\cA(\q_{\perp}(0,\beta))\}}\right] - 1. 
\end{equation}
If the diffusion is isotropic we find that
$\lambda_1=\lambda_2=\lambda_3.$ In this case we have
${\cA}_\perp(\alpha_\text{max}) = {\cA}_\perp(\alpha_\text{min}) =
B\left(\sqrt{\lambda_1}\right)$ and $\cA(\q(0,\beta_\text{max})) =
\cA(\q(0,\beta_\text{min})) = B\left(\sqrt{\lambda_1}\right)$,  and $\tau=0$.  If the diffusion is ellipsoidal and
$\lambda_2=\lambda_3$ then $\tau =
B\left(\sqrt{\lambda_2}\right)/B\left(\sqrt{\lambda_1}\right)-1>0$.
If we adopt the mixture model with multiple peaks then it is possible
to get $\tau\gg0$ even if we do not have a single diffusion PDF, and
hence we define
\begin{equation}
  \tilde{\tau} = \min_{\beta} \max_{\alpha_1,\alpha_2}
  \left\{\frac{\cA\left(\q_\perp(\alpha_1,\beta)\right)}
  {\cA\left(\q_\perp(\alpha_2,\beta)\right)}\right\} \bigg/
  \left[\frac{\cA(\q_{\perp}(0,\beta_\text{max}))}
  {\cA(\q_{\perp}(0,\beta_\text{min}))}\right] - 1.
    \label{tautilde}
\end{equation}
We note that under isotropy $\tilde\tau\equiv0$ while if we have a
single ellipsoid diffusion $\tilde\tau\equiv\tau>0$.  For a double
tensor $\tilde\tau$ is more robust and will (in general) take on a
lower value than that $\tau$ takes.  In contrast with $\tau$ and
$\tilde{\tau}$ we could also study the variability in the $q$-space
density directly in terms of the ODF.  \citet{Tuch} for examples defines the
generalized fractional anisotropy (GFA) by
\begin{equation}
  \text{GFA} = \left\{\frac{n\sum_{i=1}^{n}
      \left(\text{ODF}_W(\theta_i,\phi_i) - \frac{1}{n}\right)^2}
    {(n-1)\sum_{i=1}^n\text{ODF}_W^2(\theta_i,\phi_i)}\right\}^{1/2},
\end{equation}
and this measures the non-uniformity of the spatial distribution, as
do also the normalized entropy and the nematic order parameter \cite{Tuch}. While
the GFA quantifies the lack of uniformity in the ODF, if there are
more fibers than one, determining its statistical properties is
non-trivial, unlike the case for $\tau$ and $\tilde{\tau}$.  Another
such measure, generalized anisotropy \citep{Ozarslan2005} is defined
in terms of the generalized trace of the tensor representation of the
mean diffusivity.  
% Neither can we determine its statistical properties, nor use it to
% distinguish more than uniformity versus any alternative.

%%%%%%%%%%%%%%%%%%%%%%%%%%%%%%%%%%%%%%%%%%%%%%%%%%%%%%%%%%%%%%%%%%%%%%%%%%
\begin{figure}[!htbp]
    \begin{minipage}[]{0.32\textwidth}
      \centering
      (a)
      \includegraphics*[width=\textwidth]{pdfen12b.eps}
    \end{minipage}
    \begin{minipage}[]{0.35\textwidth}
      \centering
      (b)
      \includegraphics*[width=\textwidth]{pdfen12.eps}
    \end{minipage}\\
    \begin{minipage}[]{0.32\textwidth}
      \centering
      (c)
      \includegraphics*[width=\textwidth]{pdfen14b.eps}
    \end{minipage}
    \begin{minipage}[]{0.35\textwidth}
      \centering
      (d)
      \includegraphics*[width=\textwidth]{pdfen14.eps}
    \end{minipage}
  \caption{One- and two-dimensional summaries of Gaussian diffusion
    processes in $q$-space.  An asymmetric diffusion process is
    displayed in {\bf a} and {\bf b} (this is apparent by the
    asymmetric decay in great circles perpendicular to the dominant
    great circle in the $(\alpha,\beta)$ plane).  A scalene diffusion
    is displayed in {\bf c} and {\bf d}.  The dominant great circle is
    the solid line in the one-dimensional summaries {\bf a,c}, while
    the dotted line is the the average perpendicular diffusion over
    $\beta\in[-2,0]$ for {\bf a} and all $\beta$'s for {\bf c}.  The
    dashed line in (a) gives the average over all $\beta$'s.  In the
    two-dimensional summaries all great circles perpendicular to the
    dominant great circle are plotted on the $y$-axis to form the
    $(\alpha,\beta)$ plane.}
  \label{fig:Gaussian-PDF2}
\end{figure}
%%%%%%%%%%%%%%%%%%%%%%%%%%%%%%%%%%%%%%%%%%%%%%%%%%%%%%%%%%%%%%%%%%%%%%%%%%

\subsection{Measures of Anisotropy}

To determine the importance of the identified dominant great circle
(or orientation) we can, with a model of \eqref{ellipsoid}, compare
$B(\sqrt{\lambda_1})$ to $B(\sqrt{\lambda_2})$ and
$B(\sqrt{\lambda_3})$.  We define the following {\em anisotropy
statistic} to perform such a comparison
\begin{equation}\label{e:anisotropy}
  \xi = \frac{\log [{\cA}_\perp(0)]}{\log [{\cA}_\perp(1)]} =
  \frac{\log [B\left(\sqrt{\lambda_2}\right)]}
       {\log [B\left(\sqrt{\lambda_1}\right)]},
\end{equation}
where the last equality follows if $\lambda_3=\lambda_2$.  This
statistic measures the degree of anisotropy over the $q$-space shell
by comparing the peak-to-trough values (i.e. the value at the maximum
great circle, compared to the value at the single point perpendicular
to that maximum).  As an example Fig.~\ref{fig:Gaussian-PDF}a displays
the difference between the maximum and minimum for an average
perpendicular great circle.

The {\em decay ratio statistic} quantifies the variability of the
diffusion over the dominant great circle
\begin{equation}\label{zieq}
  \zeta =
  \max_{\beta}\frac{\log[\cA(\q(\beta))]}{\log[\cA(\q(\beta+1))]}.
\end{equation} 
When the two smaller eigenvalues ($\lambda_2$ and $\lambda_3$) are
approximately equal then $\zeta\approx1$, otherwise $\zeta\gg1$. The
scalene diffusion in Figs.~\ref{fig:Gaussian-PDF2}c
and~\ref{fig:Gaussian-PDF2}d exhibits such structure;
i.e., $\zeta\gg1$.

% To obtain some visual intuition regarding $\zeta$ we refer to
% Figs.~\ref{fig:non-Gaussian-PDF}c and~\ref{fig:non-Gaussian-PDF}d,
% where we compare the peak-to-trough values on the dominant great
% circle using our knowledge that if the variability exists on the
% dominant great circle due to two different eigenvalues in an
% ellipsoidal~PDF then these will be spaced at $\beta=1$ unit apart.
% We can then locate the axis of this variation by using the maximizer
% of \eqref{zieq}.

An indication of forking in white matter would correspond to an
asymmetric decay of the diffusion~PDF associated with different decays
depending on the parity of the deviation.  In this case we may no
longer model the diffusion~PDF as ellipsoidal.  For example, in
Figs.~\ref{fig:Gaussian-PDF2}a and~\ref{fig:Gaussian-PDF2}b we see
that while there is still a strong orientation from the dominant great
circle, the PDF no longer exhibits symmetric decay away from the
dominant great circle.  However, note that the decay is symmetric in
$\alpha$ when averaged over the full sphere to produce
${\cA}_{\perp}(\alpha)$.

Averaging over $\beta$ is no longer appropriate if we want to detect
asymmetry since a symmetric distribution will be obtained from the
Hermitian symmetry of the HARDI measurements when averaging over a
full great circle.  Instead we define a suitable {\em asymmetry
statistic} to measure potential asymmetry and we define
\begin{eqnarray}
  \kappa(\beta) &=& \frac{2\int_{0}^{1}
    \left[\cA\left(\q_\perp(\alpha,\beta)\right) -
      \cA\left(\q_\perp(-\alpha,\beta)\right)\right] \, d\alpha}
        {\frac{1}{4} \int_{0}^{4}
          \cA\left(\q_\perp(\alpha,\beta)\right) \, d\alpha},\\
        \beta_\text{max} &=& \arg \max\kappa(\beta),\\
        \kappa &=& \int_{\beta_\text{max}-1/2}^{\beta_\text{max}+1/2}
        \kappa(\beta) \, d\beta.
\end{eqnarray}
The definition of $\kappa$ is motivated by the wish to both obtain a test statistic with
sufficient power and also to reduce its variance.  The discrete
approximation to $\kappa$ will have smaller variance then
$\kappa(\beta_\text{max})$ has.  Asymmetry in the decay from the main
peak may occur when the PDF is a mixture of diffusions with varying
strengths.  If the two populations are sufficiently separated and
equivalent in magnitude then this will be indicated by $\tau$ and/or
$\tilde\tau$, and the diffusion will be recognized as a ``crossing
fiber''.  If the mixture of diffusions contains two different
strengths, then the dominating PDF will be recognized when determining
$\x_\text{max}$.  The remaining structure will not be fully consistent
with a single tensor and will (in general) appear to be asymmetric
compared to the dominant great circle.

%%%%%%%%%%%%%%%%%%%%%%%%%%%%%%%%%%%%%%%%%%%%%%%%%%%%%%%%%%%%%%%%%%%%%%%%%%
{\tiny\begin{table}[!htbp]
  \caption{The structure of the proposed models. Key to abbreviation N-P/A (Non-Preference/Anisotropy), C/E (Circular/Ellipsoidal), S/A (Symmetric/Asymmetric), I/M (Isotropic/Multimodal), M/U(Multimodal/Unimodal).}
  \label{summaryofprop}
    \begin{tabular}{lcccccc}
      \hline
      Hypothesis & Statistic & isotropic & prolate & scalene & mixture & heterogeneous\\ 
      \hline
      N-P/A & $\tau$ & small & large & large & small & large \\ 
      M/U & $\tilde{\tau}$ & small & large & large & small & large \\ 
      I/M & $\xi$ & one  & small & small & small & small\\
      C/E & $\zeta$& - & one & large & - & large\\
      S/A & $\kappa$& - & zero & zero & - & large\\
      \hline
    \end{tabular}
\end{table}}
%%%%%%%%%%%%%%%%%%%%%%%%%%%%%%%%%%%%%%%%%%%%%%%%%%%%%%%%%%%%%%%%%%%%%%%%%%

Let us discuss models that will lead to different structure in the
proposed summaries. We refer to table \ref{summaryofprop} to summarize
the properties of the summaries, and different PDFs lead to different
structure.  It may seem insufficient to consider only an isotropic
PDF, a single peak, a double peak, or something more heterogeneous.
However, even with a fully parametric model with a Gaussian DTI
framework, a two tensors model has 13 (identifiable) parameters and a
three tensor model has 19.  If one considers acquiring 60 gradient
encoding directions, then one is forced to fit a quite saturated model resulting in very noisy estimates -- especially at higher $b$-values where the
orientational heterogeneity can be well resolved.  Pushing much beyond
a small number of parameters or features of interest is not advisable
with such sampling.

\section{Estimation}
\label{matmeth}

\subsection{Parameterizing the $(\alpha,\beta)$ axes}

Having proposed various summaries of the population of PDFs at a
particular voxel, these must now be estimated from a set of diffusion
measurements.  The dominant direction is estimated by
\begin{equation}
\label{eq:lala}
\widehat{\bu}_\text{1}= \x_\text{max}=\arg \max_{ \bu,\;\|\bu\|=1}\left\{
\int_{\q\in {\cal G}(\bu)}\widehat{\cA}\left(\q\right)\;d\q
\right\}\equiv \arg \max_{\x} \text{FRT}\{\wh\cA\}(\x),
\end{equation}
where $\text{FRT}\{\cdot\}$ denotes the Funk-Radon Transform (FRT) as
utilized in \citet{Tuch}.  Note that $\wh\cA(\q)$ refers to the
multiresolution-based estimator in \citet{OlhedeWhitcher} it may be
replaced by another appropriate estimator.  We assume the availability
of $(\wh{\sigma}^{\ast})^2$ an estimator of the variance of the error
in $A(\q_k)$, namely $\sigma^2$.  The variance of $\wh\cA(\q_k)$ is
assumed to be $\tilde{\sigma}^2\le \sigma^2$ and the variance of an
interpolated value of the PDF is
$\breve{\sigma}^2\le\tilde{\sigma}^2\le \sigma^2$.  The integral may
be approximated numerically by interpolating the observed HARDI
measurements at evenly-spaced points along several great circles, each
perpendicular to some $\x_i$.  

The effects of using different numerical methods for this step is a
trade-off between increasing numerical accuracy and decreasing
variance.  Interpolating using spherical harmonics reduces variance,
but can smooth out details depending on the choice of regularization,
see also the discussion in \citet{Descoteaux} and \citet{Hess}.  We
instead use a locally linear interpolation on the polar representation
of the data, specifically enforcing the periodicity of the data.
Simple structures in terms of the observed points, can mix over
several spherical harmonics, and so the magnitude of individual
spherical harmonic coefficients may not be large, even if the local
coefficient is, thus making the representation inappropriate for using
the smoothing methods of previous authors.  The choice of
interpolation procedure should be considered in terms of which
statistic one is using, as the variance and bias must be balanced
specifically for {\em this} purpose.  We also note that spherical
harmonics do not possess the same properties as Fourier vectors, and
that an infinite number of harmonics are required for perfect
reconstruction of a surface on a sphere, and so any reconstruction from the continuous basis will be inaccurate.

The spatial maximum is determined from
$\{\text{FRT}\{\cA\}(\x_i)\}_i$.  The spatial location $\x_\text{max}$
is an {\em estimator} of $\bu_1$ and we estimate a vector in the
linear subspace spanned by $\bu_2$ and $\bu_3$ from the eigensystem of
$\bld{I}-\x_\text{max}\x_\text{max}^T$, this yielding $\hat\bu_2$ and
$\hat\bu_3$, that maximise the difference in decay in the two axes.
For numerical implementation we sample the estimated dominant great
circle by discretizing $\alpha$ to $\left\{\alpha_j\right\}_{j=1}^N$
and $\beta$ to $\left\{\beta_k\right\}_{k=1}^N$, for an even integer
$N$.  A discretized version of \eqref{qbeta} is then given by
\begin{equation}\label{e:discgreatcircle}
  \hat\q_k = \left\{
  \begin{array}{lcr}
    \beta_k \hat\bu_2 + \sqrt{1-\beta_k^2} \hat\bu_3, \, \beta_k =4k/N-1
    & \text{for} & k = 1,\dots,N/2;\\
    \beta_k \hat\bu_2 - \sqrt{1-\beta_k^2} \hat\bu_3, \, \beta_k=4k/N-3
    & \text{for} & k = N/2+1,\dots,N.
  \end{array} \right.
\end{equation}
This then allows us to discretize the estimated dominant great circle
via $\left\{\hat\q_k\right\}$.  Once $\x_\text{max}$ has been
determined we characterize the diffusion directly in $q$-space.  We
introduce additional notation by defining the sampled great circle
vectors for $\left\{\hat\q_k\right\}$ in \eqref{e:discgreatcircle} as
\begin{equation}\label{e:discperpgreatcircle}
  \hat\q_{\perp jk} = \left\{
  \begin{array}{lcr}
    \alpha_j \hat\bu_{1} + \sqrt{1-\alpha_j^2} \hat\q_{k}, \,
    \alpha_j = 4j/N-1 & \text{for} & j = 1,\dots,N/2;\\ 
    \alpha_j \hat\bu_{1} - \sqrt{1-\alpha_j^2} \hat\q_{k}, \,
    \alpha_j = 4j/N-3 & \text{for} & j = N/2+1,\dots,N.
  \end{array} \right.
\end{equation}
A discretized version of the average perpendicular diffusion
\eqref{e:averagecont} is given by $\wh\cA_\perp(\alpha_j) =
\frac{1}{N}\sum_{k}\wh\cA(\hat\q_{\perp jk})$.

\subsection{Diagnosing Non-Uniformity}

In order to test large-scale properties of the diffusion directly in
$q$-space, we consider the following statistical hypothesis
$H_0:\cA(\q)=\cA~\forall\q$ versus $H_1:\cA(\q)=\cA_E(\q)$.  Our test
statistic is based on a discretized version of \eqref{e:tau}, given by
\begin{equation}
  T = \left[\frac{\max_j\{\wh{\cA}_\perp(\alpha_{j})\}}
    {\min_j\{\cA_\perp(\alpha_{j})\}}\right] \bigg/
  \left[\frac{\max_k\{\cA(\hat{\q}_{ k})\}}
    {\min_k\{\cA(\hat{\q}_{ k})\}}\right] - 1.
\end{equation}
The distribution of this test statistic is derived in
Appendix~\ref{disttm}.  If the observations are isotropic, then the
properties along the dominant great circle will be equivalent to the
properties on the perpendicular great circle (barring
random/discretization errors).  We therefore use, with $\hat\q_k$
defined by \eqref{e:discgreatcircle}, as estimators for $\cA$ and
$\tilde\sigma$ under the null of
$\wh{\cA}(\wh{\q}_k)\cong\cA+\tilde\sigma\epsilon$,
\begin{equation}\label{e:estunderh0}
  \ol{\cA}_N = \frac{1}{N} \sum_{k=1}^N \wh{\cA}(\hat\q_k) \quad
  \text{and} \quad \hat\sigma_{\cA} = \sqrt{\rho} \,
  \text{MAD}\left\{\wh{\cA}(\hat\q_k)-\ol{\cA}_N:k=1,\dots,N\right\},
\end{equation}
where $0<\rho\le{3}$.
%\footnote{The Median Absolute Deviation (MAD) estimator of scale of a sample $\{X_k\}_{k=1}^N$ is given by $\text{MAD}\{X_k\}={\text{median}}\left\{|X_1|,\dots,|X_N|\right\}/0.6745$ \citep{Percival2000}[p.~420].}.  
These equations provide estimators of the mean value of the isotropic
diffusion and the standard deviation of $\wh{\cA}(\q)$ at the observed
measurements.  The parameter $\rho$ is a constant depending on the
linear interpolation method used for the implementation.  Taking a
value of $\rho=3$ is suitable for our choice of numerical
interpolation and we define
\begin{equation}\label{e:Um}
  U = T \frac{\overline{\cA}_N}{\hat\sigma_{\cA}}.
\end{equation}
We can compute the critical value $u_{\alpha}$ using the fact that
$F_U(u_{\alpha})=1-\alpha$, where $F_U(\cdot)$ is given by
\eqref{distributiondir}.  We report two critical values here,
$u_{0.05}=0.1185$ for the $m$ which is consistent with our sampling
scheme, and $u_{0.05}^{(\text{con})}=1.9637$ with a conservative
distribution approximation.

% Using simulation studies to understand the PDF of $T$ will not work
% in most cases in this scenario (see \citet{ale-bar-arr:detection} for
% such simulation studies of similar statistics), as the parameters of
% the distribution of $T$ depend on the true, but unknown, values of
% $\cA$ and $\cA_\text{min}$.  Given a sufficiently large sample from
% a population of homogeneous voxels it is possible to use
% simulation-based methods to separate voxels coming from a single
% prolate diffusion~PDF versus multiple tensors.  However, such a
% data-driven approach is not appropriate in practice when only a
% single voxel is available to characterize any given
% population. Distributions arrived at from different parameter
% settings cannot be easily generalized.  The test statistic $U$
% defined in \eqref{e:Um} has, to first order, a distribution under
% $H_0$ that does not depend on any unknown parameters -- this making
% it a valid statistic.

We also develop a new test based on a null of a multi-modal diffusion,
where we define multi-modal in terms of
$(\wt{\cA}_{\text{max}}\cA_\text{min}) /
(\wt{\cA}_{\text{min}}\cA_\text{max})<c=2$, say, where
$\wt{\cA}_{\text{max}}$ and $\wt{\cA}_{\text{min}}$ are the maximum
and minimum on the perpendicular great circle minimizing
\eqref{tautilde} in $\beta$, respectively, while $\cA_{\text{max}}$
and $\cA_{\text{min}}$ are the maximum and minimum on the dominant
great circle.  The level $c$ is fairly arbitrary, but to develop a
powerful method of separating the clearly unimodal from the
multi-modal some level must be chosen, based on the deterministic
structure of the sampled PDF.  To separate the unimodal from the
multimodal we start from $\tilde{\tau}$ and define
\begin{equation}
  \wt{T} = \min_{k} \left\{\max_{{j_1},{j_2}} 
  \left\{\frac{\wh{\cA}\left(\hat\q_{\perp j_1 k}\right)}
              {\wh{\cA}\left(\hat\q_{\perp j_2
  k}\right)}\right\}\right\}
              \bigg/ \left[\frac{\max_k\{\wh{\cA}(\hat\q_{k}))\}}
                {\min_k\{\cA(\hat\q_{k})\}}\right] - 1.
\end{equation}
We shall now choose to distinguish the multimodal from the unimodal,
and so normalize by $\wt{U} = (\wt{T}-[c-1])
\wh{\cA}_{\text{min}}/(\hat\sigma_{\cA}\sqrt{2c^2+2})$, where
$\wh{\cA}_{\text{min}}=\wh{\cA}(\wh{\bm{\upsilon}}_1)$.  The
distribution of this test statistic is derived in
Appendix~\ref{disttm}, under the specified null hypothesis.

If on the other hand we have failed to reject the null hypothesis
``$\cA(\q_{\perp}(\alpha,\beta))$ {\em equally} variable in $\beta$
for $\alpha=0$ as it is in $\alpha$ for a fixed $\beta$,'' then based
on the $T$-statistic we need to distinguish voxels that indicate two
fiber populations versus isotropic voxels.  We define a discrete
version of \eqref{e:anisotropy} to be
\begin{eqnarray}\label{e:degreeofanisotropy}
  X = \frac{\log[\wh{\cA}_\perp(0)]}{\log[\wh{\cA}_\perp(1)]}.
\end{eqnarray}
We can interpret $\xi$, and the sample version $X$, as the degree of
anisotropy from the average perpendicular great circle.  The ADC is
defined at a set of defined $\q$ vectors $\tilde{\q}_j$ as
$\wh{C}(\tilde\q_j)=-b^{-1}\log[{A(\tilde\q_j)}]$
\citep{ale-bar-arr:detection}.  We recognize that the statistic $X$ is
comparing the average ADC on the great circle to the average ADC
perpendicular to the great circle, or that
\eqref{e:degreeofanisotropy} may be re-written in terms of the ADC at
a fixed value of $b$ via
\begin{equation}
  X = \sum_k\wh{C}(\hat\q_k)/\sum_k\wh{C}(\hat\q_{\perp N/4 k})).
\end{equation}
With an assumption of ellipsoidal structure, {\em cf}
\eqref{ellipsoid}, we have averaged the ADC to reduce variance when
estimating $\xi$ without accruing bias.  We define $X_k$ as the sample
anisotropy calculated using only the $k$th perpendicular great circle,
or
$X_k=\log[\wh{\cA}(\hat\q_{k})]/\log[\wh{\cA}(\hat\q_{\perp{N/4}k})]$,
and refer to \eqref{e:discperpgreatcircle} for the definition of
$\hat\q_{\perp jk}$.  Under moderate-to-high SNR we may approximate
this as a Gaussian random variable.  We quantify uncertainty, when
there are potentially several peaks, using
$\hat\sigma_2=\min\{\hat\sigma_{\cA},\hat\sigma^{\ast}\}$, where
$\hat\sigma_{\cA}$ is defined in \eqref{e:estunderh0} and
$\wh{\sigma}^{\ast}$ is the available estimator for $\sigma$.  By
using the minimum we ensure that the variance is never estimated as
greater than before smoothing.

For those voxels where we do not reject isotropy, we may now
distinguish between isotropy and a multiple-tensor model using the
$X$.  We define
\begin{equation}
 Q = \frac{\rho(X-1)}{\hat\sigma_{2}}
  \left|\overline{\cA}_N\log\overline{\cA}_N\right|
\end{equation}
as the test statistic for multimodality. So we consider the test
$H_0:\cA(\q)=\cA~\forall\q$ versus
$H_1:\max_k\{\cA(\q_{k})\}\gg\min_k\{\cA(\q_{k})\}$ (multiple peaks)
and use $Q$ as the test statistic, whose distribution under the null
is found in Appendix~\ref{threshxi}.  These three tests allow us to at
a single voxel diagnose the structure of the PDF, where $U$ is used to
separate anisotropic PDFs from isotropic PDFs, $\wt{U}$ is used to
separate ellipsoid PDFs from multimodal PDFs and $Q$ is used to
separate multimodal from isotropic PDFs.

\subsection{Diagnosing Asymmetry}

Having established methodology to discriminate the number of peaks in
the diffusion~PDF at a single voxel, we now provide additional
methodology to characterize the diffusion~PDF as scalene versus other
forms of asymmetry; e.g., to observe an indication of forking.  We
start by determining
\begin{equation}
  k_\text{max} = \arg\max_{1\le k\le N/4}
  \frac{\log\wh{\cA}(\hat{\q}_k)}{\log\wh{\cA}(\hat{\q}_{k+N/4})}.
\end{equation} 
We define the parameters $m<m'<2m$ (recall that $m$ is the effective degrees
of freedom) for robustness
so that
\begin{equation}
\label{defofZ}
  Z =\frac{\log\wh{\cA}\left(\hat{\q}_{{k}_\text{max}+N/(2m')}\right)}
  {\log\wh\cA\left(\hat{\q}_{{k}_\text{max}+N/(2m')+N/4}\right)}.
\end{equation}
We remark that $Z$ is related in some sense to $X_k$ (refer to Figs.
\ref{fig:Gaussian-PDF} and \ref{fig:Gaussian-PDF2}); $X_k$ compares
the value of the diffusion on location $k$ on the dominant great
circle ($\alpha=0$) to the value at the perpendicular to the dominant
great circle ($\alpha\neq 0$), whereas $Z$ in contrast looks at the
difference in values of the diffusion on the great circle itself
($\alpha=0$ and $\beta$ varies).  Under the null hypothesis
$\cA(\cdot)$ is constant on the great circle, and if the medium and
minor eigenvalues are approximately equal then
$\E\{Z\}=\zeta\approx{1}$, otherwise $\zeta\gg1$.  We define a
normalized version of the decay ratio statistic \eqref{defofZ}, given
by
\begin{equation}
  V = \frac{(Z-1)\left|\overline{\cA}_N\log\overline{\cA}_N\right|}%
  {\hat\sigma_2} .
\end{equation}
A suitable threshold for this statistic may be found in
Appendix~\ref{threshzeta}.  The statistics $Q$ and $V$ used to test
different hypotheses of non-isotropic decay, have similar forms.

The summary $\kappa$ allows us to diagnose structure inconsistent with
a single ellipsoid diffusion.  Departures from a single ellipsoid
diffusion structure could be attributable to partial volume effects,
or a heterogeneous population of fibers.  For such a model,
\eqref{ellipsoid} is no longer appropriate and we would rather fit a
mixture model with unequal populations; or possibly
$\cA_{DE}\left(\cdot \right)$.  In such circumstances one cannot use
the average perpendicular great circle to uncover asymmetry since
averaging over all possible $\beta$'s will produce a symmetric
distribution regardless of the underlying fiber characteristics.
Taking $\breve{k}_\text{max}=\arg\max_k{P_k}$ we define the {\em
  asymmetry statistic} by
\begin{equation}
  K = \frac{1}{N/4+1}
  \sum_{k=\breve{k}_\text{max}-N/8}^{\breve{k}_\text{max}+N/8} P_k,\;
  P_k = \frac{8\sum_{j=1}^{N/4-1} \left[\wh{\cA}(\hat\q_{\perp jk}) -
        \wh{\cA}(\hat\q_{\perp(j+N/4)k})\right]} {\sum_{j=1}^N
        \wh{\cA}(\hat\q_{\perp jk})},
\end{equation}
Full details on the distribution of this test statistic may be found
in Appendix~\ref{threshkappa}.  We have chosen $N/8$ to improve the
power -- averaging decreases the variance, but the asymmetry is
greatest near the maximum (compare with Figure
\ref{fig:Gaussian-PDF2}(b)).  For tests at a specific voxel we perform
the hypothesis test $H_0:\kappa=0$ versus $H_1:\kappa\neq0$, using
quantiles from the standard Gaussian~PDF $\phi(\cdot)$.  This
identifies diffusion~PDFs that are non-Gaussian in terms of the parity
structure in the principal axes.  However, it does not compare the
maximum and minimum of a perpendicular great circle, rather it finds a
set of perpendicular great circles for which the decay around the
dominant great circle is asymmetric and estimates this average
asymmetry; e.g., Figure~\ref{fig:Gaussian-PDF2}(b).

% It is also possible to extend these methods to multiple shells
% instead of typical HARDI sampling \citep{Wu2007}.  In this case the
% statistics are calculated for each shell, and averaged over the
% different shells. The dominant orientation would be estimated by a
% weighted averaging of the estimated dominant orientations for each
% shell, as its distribution is SNR dependent, which changes at each
% shell.

The usage of the statistics is now combined at a voxel level. The most
important characteristic is to classify the voxel as isotropic, single
peak, or multiple peak. With this information the local structure of
the peaks can be further characterised at a local level: when
comparing the PDFs between voxels for tracking first the number of
voxels is important, and after the mixture components is matched to
the most appropriate component from its local characteristics, we can
anticipate varying asymmetry values before forking structure; as shall
be further discussed in the applications section.

\subsection{Example: Crossing and Forking Fibers}

We consider two typical heterogeneous white-matter structures, a
crossing fiber and a forking fiber in Fig.~\ref{fig:evolution}.  The
spatial representation of the forking fiber is the first row,
enumerated by (i), while the $q$-space representation of the forking
fiber is on the second row, enumerated by (ii).  In the spatial
representation, we see a single fiber population in voxel~(i,a) and as
we traverse from left-to-right the two populations become more
apparent until a second fiber appears in voxel~(i,g).  The $q$-space
version of these two populations shows a scalene distribution
developing in voxels~(ii,b)--(ii,e).  As the forking progresses from
left-to-right it appears both highly warped and scalene until the
distribution clearly displays multiple fibers in voxel~(iii,g).  These
fibers were generated by aggregating two densities or
\begin{equation}\label{example-fibers}
  \cA(\q) = a_1(t) \cA_g(\q; \bm{\Lambda}^{(1)}(t),
  \bm{\Upsilon}^{(1)}(t))+(1-a_1(t)) \cA_g(\q; \bm{\Lambda}^{(2)}(t),
  \bm{\Upsilon}^{(2)}(t)).
\end{equation}
In the case of the forking fiber $a_1(t)=1-t$ and the angle between
the principal direction of $ \bm{\Upsilon}^{(1)}(t))$ and $
\bm{\Upsilon}^{(2)}(t))$ $\pi{t/2}$, with the individual tensors
taking values similar to $\cA_1(\q)$ and $t\in[0,1]$.  The spatial
representation of the crossing fiber is the third row (iii), with its
corresponding $q$-space representation in the fourth row (iv).  The
ellipsoid appears prolate in voxel~(iii,a), then two fiber populations
are present in voxel~(iii,d) and eventually the fiber population
returns to a prolate shape.  With respect to the parameterization of
the crossing fiber in \eqref{example-fibers},
$a_1(t)\in\{0,0.25,0.5\}$ and the two fibers cross at 90~degrees with
parameters similar to $\cA_1(\q)$.

The description of the crossing fiber is in many ways simpler than the
forking.  Table~\ref{raw_properties} lists the summary statistics
($\tau,\tilde\tau,\xi,\zeta,\kappa$) for the crossing and forking
fiber examples in Fig.~\ref{fig:evolution}.  Note that these
deterministic summaries have not been normalized, unlike the
statistics in Section~\ref{matmeth} (as there is in this case no noise variance to compare with).  The mixture of populations of
unequal strength in the forking fiber shows a number of
characteristics not found in the crossing fiber.  For example, the
forking fiber is clearly diagnosed as a single population until
voxels~(i,e), (i,f) and (i,g), where there is increasing heterogeneity
in the fiber population.  This is exhibited by increasing values for
the decay ratio $\zeta$-statistic, and the asymmetry
$\kappa$-statistic.  For the crossing fiber, we clearly detect the
multiple-fiber population in voxel~(iii,d) using either the $\tau$ and
$\tilde\tau$ statistics.  The multiple-fiber characteristics in the
forking example are more complex, where the second fiber population is
initially dominated by the first.  If we examine the crossing fiber
more closely there is little apparent asymmetry and we can compare the
asymmetry statistic, where $\kappa\approx{0}$ versus
$0.15\leq{\kappa}\leq0.45$ for voxels~(i,c)--(i,e).  To distinguish
the clear multi-tensors from uniformity we observe that $\xi<1$, which
is the expected value under the hypothesis of isotropy.


%%%%%%%%%%%%%%%%%%%%%%%%%%%%%%%%%%%%%%%%%%%%%%%%%%%%%%%%%%%%%%%%%%%%%%%%%%
\begin{figure}[!htbp]
(i)
    \begin{minipage}[]{0.12\textwidth}
      \centering
      (a)
      \includegraphics*[width=\textwidth]{figure1a1.eps}
    \end{minipage}
    \begin{minipage}[]{0.12\textwidth}
      \centering
      (b)
      \includegraphics*[width=\textwidth]{figure1e1.eps}
    \end{minipage}
    \begin{minipage}[]{0.12\textwidth}
      \centering
      (c)
      \includegraphics*[width=\textwidth]{figure1f1.eps}
    \end{minipage}
    \begin{minipage}[]{0.12\textwidth}
      \centering
      (d)
      \includegraphics*[width=\textwidth]{figure1g1.eps}
    \end{minipage}
    \begin{minipage}[]{0.12\textwidth}
      \centering
      (e)
      \includegraphics*[width=\textwidth]{figure1h1.eps}
    \end{minipage}
    \begin{minipage}[]{0.12\textwidth}
      \centering
      (f)
      \includegraphics*[width=\textwidth]{figure1i1.eps}
    \end{minipage}
    \begin{minipage}[]{0.12\textwidth}
      \centering
      (g)
      \includegraphics*[width=\textwidth]{figure1j1.eps}
    \end{minipage}
    \\(ii)
    \begin{minipage}[]{0.12\textwidth}
      \centering
      \includegraphics*[width=\textwidth]{figure2b1.eps}
    \end{minipage}
    \begin{minipage}[]{0.12\textwidth}
      \centering
      \includegraphics*[width=\textwidth]{figure2f1.eps}
    \end{minipage}
    \begin{minipage}[]{0.12\textwidth}
      \centering
      \includegraphics*[width=\textwidth]{figure2g1.eps}
    \end{minipage}
    \begin{minipage}[]{0.12\textwidth}
      \centering
      \includegraphics*[width=\textwidth]{figure2h1.eps}
    \end{minipage}
    \begin{minipage}[]{0.12\textwidth}
      \centering
      \includegraphics*[width=\textwidth]{figure2i1.eps}
    \end{minipage}
    \begin{minipage}[]{0.12\textwidth}
      \centering
      \includegraphics*[width=\textwidth]{figure2j1.eps}
    \end{minipage}  
    \begin{minipage}[]{0.12\textwidth}
      \centering
      \includegraphics*[width=\textwidth]{figure2k1.eps}
    \end{minipage}\\
    (iii)
    \begin{minipage}[]{0.12\textwidth}
      \centering
      \includegraphics*[width=\textwidth]{figure3a1.eps}
    \end{minipage}
    \begin{minipage}[]{0.12\textwidth}
      \centering
      \includegraphics*[width=\textwidth]{figure3a1.eps}
    \end{minipage}
    \begin{minipage}[]{0.12\textwidth}
      \centering
      \includegraphics*[width=\textwidth]{figure3b1.eps}
    \end{minipage}
    \begin{minipage}[]{0.12\textwidth}
      \centering
      \includegraphics*[width=\textwidth]{figure3c1.eps}
    \end{minipage}
    \begin{minipage}[]{0.12\textwidth}
      \centering
      \includegraphics*[width=\textwidth]{figure3b1.eps}
    \end{minipage}
    \begin{minipage}[]{0.12\textwidth}
      \centering
      \includegraphics*[width=\textwidth]{figure3a1.eps}
    \end{minipage}
    \begin{minipage}[]{0.12\textwidth}
      \centering
      \includegraphics*[width=\textwidth]{figure3a1.eps}
    \end{minipage}\\
    (iv)
    \begin{minipage}[]{0.12\textwidth}
      \centering
      \includegraphics*[width=\textwidth]{figure4a1.eps}
    \end{minipage}
    \begin{minipage}[]{0.12\textwidth}
      \centering
      \includegraphics*[width=\textwidth]{figure4a1.eps}
    \end{minipage}
    \begin{minipage}[]{0.12\textwidth}
      \centering
      \includegraphics*[width=\textwidth]{figure4b1.eps}
    \end{minipage}
    \begin{minipage}[]{0.12\textwidth}
      \centering
      \includegraphics*[width=\textwidth]{figure4c1.eps}
    \end{minipage}
    \begin{minipage}[]{0.12\textwidth}
      \centering
      \includegraphics*[width=\textwidth]{figure4b1.eps}
    \end{minipage}
    \begin{minipage}[]{0.12\textwidth}
      \centering
      \includegraphics*[width=\textwidth]{figure4a1.eps}
    \end{minipage}
    \begin{minipage}[]{0.12\textwidth}
      \centering
      \includegraphics*[width=\textwidth]{figure4a1.eps}
    \end{minipage}
  \caption{An illustration of the evolution of a diffusion~PDF through
    a number of adjacent voxels.  The first and second rows are the
    spatial and $q$-space evolution, respectively, of the
    diffusion~PDF for a forking fiber.  The third and fourth rows are
    the spatial and $q$-space evolution, respectively, of the
    diffusion~PDF for a crossing fiber.}
  \label{fig:evolution}
\end{figure}
%%%%%%%%%%%%%%%%%%%%%%%%%%%%%%%%%%%%%%%%%%%%%%%%%%%%%%%%%%%%%%%%%%%%%%%%%%

%%%%%%%%%%%%%%%%%%%%%%%%%%%%%%%%%%%%%%%%%%%%%%%%%%%%%%%%%%%%%%%%%%%%%%%%%%
\begin{table}[!htbp]
  \caption{Discretized statistics based on non-parametric measures of
    symmetry for a simulated forking and crossing fiber, compare with
    Table~\ref{summaryofprop}.}
  \label{raw_properties}
  \begin{tabular}{cccccccc}
    \hline
Statistic        & \multicolumn{7}{c}{Forking fiber}\\
\cline{2-8}
                 & (i,a) & (i,b) & (i,c) & (i,d) & (i,e) & (i,f) & (i,g)\\
\hline
$\tau$           & 10.18  & 5.12 & 3.35 & 1.86 & 0.82 & 0.07 & -0.14\\
$\tilde{\tau}$   & 8.98  & 4.94 & 3.36 & 2.13 & 1.19 & 0.33 & -0.07\\
$\xi$            & 0.12  & 0.18 & 0.21 & 0.27 & 0.35 & 0.53  & 0.69\\
$\zeta$          & 1.03  & 1.51 & 1.72 & 1.91 & 2.06 & 2.21  & 2.67\\
$\kappa$         & -0.01 & 0.03 & 0.17 & 0.35 & 0.45 & 0.54  & 0.30\\
\hline
                 & \multicolumn{6}{c}{Crossing fiber}\\
\cline{2-8}
                 & (iii,a) & (iii,b) & (iii,c) & (iii,d) & (iii,e) & (iii,f)\\
\hline
$\tau$           & 9.19 & 9.19 & 1.15 & -0.14 & 1.15 & 9.19\\
$\tilde\tau$     & 8.98 & 8.98 & 1.26 & -0.07 & 1.26 & 8.98\\
$\xi$            & 0.12 & 0.12 & 0.32 &  0.69 & 0.32 & 0.12\\
$\zeta$          & 1.04 & 1.04 & 1.74 &  2.67 & 1.74 & 1.04\\
$\kappa$         & 0.00 & 0.00 & 0.09 &  0.30 & 0.09 & 0.00\\
\hline
  \end{tabular}
\end{table}
%%%%%%%%%%%%%%%%%%%%%%%%%%%%%%%%%%%%%%%%%%%%%%%%%%%%%%%%%%%%%%%%%%%%%%%%%%

\section{Simulation Study}

We illustrate the properties of the proposed $q$-space summary
statistics for the diffusion~PDF on a variety of simulated diffusions
processes.  The following six models attempt to cover common, and not
so common, diffusion processes that include both unimodal and multiple
tensors
{\small \begin{eqnarray}
\label{e:A5}
  \cA_1(\q) &=& \exp(-t \q^T \wt{\bld{D}}_i \q), \; i=1,2,3,\\
  \wt{\bld{D}}_1 &=& 68 \tilde\be_1 \tilde\be_1^T + 8\tilde\be_2 
  \tilde\be_2^T + 8\tilde\be_3 \tilde\be_3^T,\nonumber\\ 
  \wt{\bld{D}}_2 &=& 68 \tilde\be_1 \tilde\be_1^T + 15\tilde\be_2 
  \tilde\be_2^T + \tilde\be_3 \tilde\be_3^T,\nonumber\\
  \wt{\bld{D}}_3 &=& 28\tilde\be_1 \tilde\be_1^T + 28\tilde\be_2 
  \tilde\be_2^T + 28\tilde\be_3 \tilde\be_3^T, \nonumber\\
  \cA_4(\q) &=& 0.5 \exp(-t \q^T \wt{\bld{D}}_1 \q) + 0.5 \exp(-t \q^T
  \wt{\bld{D}}_4 \q),\\
  \wt{\bld{D}}_4 &=& 68\tilde\be_2 \tilde\be_2^T + 8\tilde\be_1 
  \tilde\be_1^T + 8\tilde\be_3 \tilde\be_3^T,\nonumber\\
  \cA_5(\q) &=& \exp\left(-11t |\q^T\tilde\be_2|^2\right)
  \Big|\exp\left(-68t |\q^T\tilde\be_1|^2\right)\\
  && \times \left[\exp\left(-0.2t |\q^T\tilde\be_3|^2\right) + 
    \exp(-35t |\q^T\tilde\be_3|^2)\right] \nonumber\\ \nonumber
  && + \frac{4}{\pi}D\left(\sqrt{68t} \q^T \tilde\be_1\right)
  \left[D\left(\sqrt{35t} \q^T \tilde\be_3\right) -
  D\left(\sqrt{0.2t}\q^T \tilde\be_3\right)\right]\Big|, \\
  \cA_6(\q) &=&0.3\exp(-t \q^T \wt{\bld{D}}_1 \q)+0.7
  \exp(-t \q^T \wt{\bld{D}}_5 \q),\\
  \wt{\bld{D}}_5 &=& 42.5\breve\be_1\breve\be^T_1 + 14\breve\be_2 
  \breve\be^T_2 + 20\breve\be_3 \breve\be^T_3, \nonumber
\end{eqnarray}}
where $D(x)=\exp(-x^2)\int_0^x\exp(t^2)\,dt$ is the Dawson function
\citep{abra}.  We define $\tilde\be_j=\R\be_j$, where the matrix $\R$
rotates the axes $(\be_1,\be_2,\be_3)$ to a new coordinate system
$(\tilde\be_1,\tilde\be_2,\tilde\be_3)$.  This extra step is added to
protect against systematic bias in our estimation procedure due to the
diffusion~PDF coinciding with the sampling grid.  In $\cA_6(\q)$ this
rotation is not implemented, but
$(\breve\be_1,\breve\be_2,\breve\be_3)$ has been rotated with respect
to $(\be_1,\be_2,\be_3)$ to produce an asymmetric diffusion in the
multiple-tensor model.

We have chosen to define $\wt{\bld{D}}_i=4\times10^{10}\bld{D}_i$,
$i=1,\dots,4$, and normalized $\|\q\|=1$.  With $t=0.04$ this
corresponds to $b=4t\times10^{10}=1600\,\text{s/mm}^2$
\citep{Alexander2005} and the trace of the first three non-normalized
matrices $\bld{D}_i$ as $2.1\times10^{-9}\,\text{m$^2$/s}$.  The
function $\cA_5(\q)$ is obtained from the magnitude of the FT of an
asymmetrically decaying diffusion process in space.  We illustrate a
range of behavior for the scalar statistics defined in $q$-space using
these test functions, providing only a subset in order to compare and
contrast their performance.  We simulate 1000 realizations for each
test function and add Gaussian noise with standard deviation of
$\cA(0)/2$, $\cA(0)/10$, $\cA(0)/20$ and $\cA(0)/30$ to both the real
and imaginary channels using 60 gradient directions.

%%%%%%%%%%%%%%%%%%%%%%%%%%%%%%%%%%%%%%%%%%%%%%%%%%%%%%%%%%%%%%%%%%%%%%%%%%
{\tiny\begin{table}[!htbp]
  \caption{Hypothesis tests for the six diffusion processes
    $\{\cA_i\}_{i=1}^6$.  The nominal size of the tests is $5\%$ for
    $U$, $V$ and $Q$, while the nominal size is $10\%$ for $K$ and
    $\wt{U}$.  The tests have been carried out at different SNRs. Key to abbreviation N-P/A (Non-Preference/Anisotropy), C/E (Circular/Ellipsoidal), S/A (Symmetric/Asymmetric), I/M (Isotropic/Multimodal), M/U (Multimodal/Unimodal). The SNR decreases the further down the entries are in the table, ranging from SNR=$\frac{1}{30}$, SNR=$\frac{1}{20}$, SNR=$\frac{1}{10}$ to SNR=$\frac{1}{2}$.}
    \begin{tabular}{lccccccc}
      \hline
      Hypothesis & Statistic & $\cA_1$ & $\cA_2$ & $\cA_3$ & $\cA_4$ & $\cA_5$ & $\cA_6$\\ 
\hline
$H_0/H_1,$  & & prolate & scalene & isotropic & crossing & asym. & asym.\\
\hline
N-P/A     & $U$ &  1000 & 988  & 26 & 492 & 1000 & 802\\ 
C/E & $V$ & 146/1000 & 924/988 & 0/26 & 382/492 & 120/1000 & 495/802\\
S/A      & $K$ & 191/1000 & 108/988 & 10/26 & 258/492 & 491 /1000&
      208/802 \\ 
I/M    & $Q$ & 0/0 & 12/12 & 21/974  & 420/508 & 0/0 & 195/198\\ 
M/U       & $\wt{U}$ & 991/1000 & 136/988 & 0/26 & 239/492 &  996/1000 & 19/802\\
      \hline
$H_0/H_1,$  & & & & & & & \\
\hline
N-P/A   & $U$ & 1000 & 855 & 26 & 484 & 1000 & 727\\ 
C/E & $V$ & 153/1000 & 794/855 & 0/26 & 338/484 & 148/1000 & 267/727\\
S/ & $K$ & 148/1000 & 38/855 & 10/26 & 199/484 & 331/1000 & 136/727\\ 
I/M& $Q$ & 0/0 & 0/145 & 23/974  &259/516& 0/0 & 201/273\\ 
M/U & $\wt{U}$ & 945/1000 & 46/805 & 0/23   &   151/484 & 942/1000  & 8/727\\
      \hline
$H_0/H_1,$  & & & & & & & \\
\hline
N-P/A   & $U$ & 1000 & 239 & 34 & 441 & 998 & 457\\ 
C/E & $V$ & 225/1000  & 214/239 & 0/34 & 192/441 & 194/998 & 58/457\\
S/A & $K$ & 89/1000   & 1/239 & 10/34 & 99/441 & 139/998 & 74/457\\ 
I/M& $Q$ & 0/0 & 449/761 & 20/966  & 20/539&1/2  &18/543 \\ 
M/U & $\wt{U}$ & 239/1000 & 7/239 & 0/34 & 27/441 & 174/998  & 4/457\\
      \hline
$H_0/H_1,$  & & & & & & & \\
\hline
N-P/A   & $U$ & 45 & 21 & 25 & 31 & 37 & 24\\ 
C/E & $V$ & 2/45 & 2/21 & 5/25 & 6/31 & 4/37 & 4/24\\
S/A & $K$ & 5/45 & 4/21 & 2/25 & 1/31 & 4/37 & 3/24\\ 
I/M& $Q$ & 1/955 & 3/979 & 2/975  &2/969& 4/963 &5/976 \\ 
      M/U & $\wt{U}$ & 2/45 & 0/21 & 1/25 &0/31  & 5/37  & 1/24\\
      \hline
      \hline
    \end{tabular}
  \label{testing}
\end{table}}
%%%%%%%%%%%%%%%%%%%%%%%%%%%%%%%%%%%%%%%%%%%%%%%%%%%%%%%%%%%%%%%%%%%%%%%%%%

Results, provided in Table~\ref{testing}, are consistent with varying
degrees of the SNR.  The prolate diffusion $\cA_1$ is clearly
detectable, down to an $\text{SNR}=1/10$, despite using non-parametric
methods via the $U$-statistic.  Detecting the scalene diffusion
depends on the SNR, while the isotropic diffusion is clearly
distinguishable from its alternatives under the full range of SNR
using the $U$-statistics.  The multi-tensor diffusion $\cA_4$ is
difficult to classify using the $U$-statistics and its correct
classification depends on how well the location of the dominant peak
is estimated.  If the dominant peak is well determined then the
$U$-statistic clearly recognizes the density as anisotropic, if not
the $q$-space measurements are characterized as non-Gaussian instead
of multimodal.  If one was only concerned with empirically separating
prolate diffusion~PDFs from multimodal diffusion~PDFs, rather than
performing a hypothesis test, then this would be relatively
straightforward; e.g., retaining 95\% of the unimodal Gaussian with
the $\text{SNR}=1/20$ leads to rejecting all but 11\% of the
multi-tensor realizations (see the $\tilde{U}$-statistic).  Since we
are interested in detecting ellipsoidal decay around a single
direction, the variation over the dominant great circle will be large
for anisotropic voxels with ellipsoidal decay as well as for
multi-modal diffusion~PDFs.  At an $\text{SNR}=1/20$ the
$\wt{U}$-statistic provides complimentary information by strongly
separating the prolate (94\% rejected) from the multi-tensor model
(15.1\% rejected, near the nominal value of 10\%), but fails to
distinguish between the scalene and the multi-tensor models
(Table~\ref{testing}).  The highly scalene density is instead seen as
multimodal, and such structure may be approximated using two tensors,
especially when sparsely sampled on the sphere.

The two distributions with constant behavior on the dominant great
circle are not diagnosed with asymmetric decay, while the null
hypothesis is rejected for $\cA_2$ in a substantial number of cases in
Table~\ref{testing}.  The misdiagnosed multimodal diffusion~PDF
$\cA_4$ also has the null hypothesis of multimodality rejected for a
substantial number of cases.  This is to be expected since the
observed diffusion will experience considerable variation over the
dominant great circle, consistent with observing a diffusion process
with a single dominant direction and ellipsoidal decay.

We fail to reject the null hypothesis of symmetry for the two
diffusion processes that are symmetric ($\cA_1$ and $\cA_2$) in most
cases, while we reject a larger proportion for $\cA_5$.  There is
unfortunately a lack of power in this test which is due to sampling
60~directions; limiting the performance of the test statistic.  For
$\cA_6$ and keeping the $\text{SNR}=1/20$, we reject the null
hypothesis 38.9\% with 60~directions.  For $\cA_5$ we reject the null
51.6\% of the time using 245~directions at $\text{SNR}=1/20$ -- a
clear increase from 35.2\% with 60~directions.  Increasing the SNR
also increases our power to detect such asymmetry, as shown in
Table~\ref{testing}.  The power of the test improves as the number of
directions increase or the amount of asymmetry (better characterised
with better spherical sampling) increases.

\section{Analysis of Clinical Data}

HARDI data were acquired from one normal subject (30 year old,
male~Caucasian) in a Siemens TIM Trio 3.0~Tesla scanner using a
32-channel head coil.  
% Measurement of 64 gradient directions ($b=1600~\text{s/mm}^2$) and
% one T2 image ($b=0$) were obtained using a twice refocused diffusion
% preparation.  The slice prescription was: 64~slices acquired in the
% AC-PC plane, $\text{TE}=95~\text{ms}$,
% $\text{FoV}=240\times240~\text{mm}$, $\text{base
% resolution}=128\times128$, slice thickness of $1.9~\text{mm}$ and
% cardiac gating was applied.
Regions of interest (ROIs) from two slices of the clinical data are
provided to illustrate the statistical summaries developed in this
paper.  Slice~1 contains an ROI that is dominated by single-fiber
voxels containing structures such as the corpus callosum and cingulum.
Figure~\ref{figure-slice1}(a) shows the voxels using the common
color-coding convention (i.e., RGB for the $(x,y,z)$ coordinates)
weighted by the estimated fractional anisotropy (FA) at each voxel.
The FA for the ROI is reproduced in Figure~\ref{figure-slice1}(b)
along with the $p$-values for the anisotropy and ellipsoidality
statistics in Figures~\ref{figure-slice1}(c)
and~\ref{figure-slice1}(d), respectively.  We select a very liberal
threshold ($p=0.15$) for the purpose of exploratory data analysis, not
confirmatory data analysis.  We observe very few voxels that indicate
asymmetry at specific voxels, while the ellipsoidality $p$-values
indicate quite a few voxels that exhibit ellipsoidal structure.  These
voxels are located at the borders of strongly directional structures
such as the corpus collosum and cingulum, and reaffirms the results
obtained in the simulation studies.  Additional information about the
structure is obtained by plotting the test statistic for unimodality
and the $p$-values from the multi-modality test statistic in
Figures~\ref{figure-slice1}(e) and~\ref{figure-slice1}(f),
respectively.  The corpus callosum, and to a lesser extent the
cingulum, produce large values in the unimodality test statistic as to
be expected from those structures.  Multimodality is detected in
voxels with reduced FA and/or on the edges of prominent white-matter
structures.  The pattern of multimodal voxels identified in
Figure~\ref{figure-slice1}(f) in general do not appear to overlap with
those voxels that were identified using the ellipsoidality statistic,
providing evidence that this methodology is detecting distinct
features in the white-matter microstructure.

%%%%%%%%%%%%%%%%%%%%%%%%%%%%%%%%%%%%%%%%%%%%%%%%%%%%%%%%%%%%%%%%%%%%%%%%%%
\begin{figure}[!htbp]
    \begin{minipage}[]{0.45\textwidth}
      \centering
      \includegraphics*[width=\textwidth]{slice1_slice+rect.eps}
    \end{minipage}
    \begin{minipage}[]{\textwidth}
      \centering      
      \includegraphics*[width=.85\textwidth]{slice1_statistics.eps}
    \end{minipage}
  \caption{Axial slice from clinical HARDI acquisition.  Color-coded
    fractional anisotropy (FA) for the whole slice is displayed in
    {\bf (a)} along with the boundaries for the ROI.  For the
    zoomed-in ROI: color-coded FA {\bf (b)}, anisotropy $p$-values
    {\bf (c)}, ellipsoidality $p$-values {\bf (d)}, unimodality test
    statistic {\bf (e)} and multimodality $p$-values {\bf (f)}.
  \label{figure-slice1}}
\end{figure}
%%%%%%%%%%%%%%%%%%%%%%%%%%%%%%%%%%%%%%%%%%%%%%%%%%%%%%%%%%%%%%%%%%%%%%%%%%

The ROI selected in slice~2 captures more complicated interactions
between white-matter structures such as the corticopontine tract,
anterior thalamic radiation and corpus callosum;
Figure~\ref{figure-slice2}(a).  The FA for the ROI is reproduced in
Figure~\ref{figure-slice2}(b) along with the $p$-values for the
anisotropy and ellipsoidality statistics in
Figures~\ref{figure-slice2}(c) and~\ref{figure-slice2}(d),
respectively.  Asymmetry is difficult to detect in these data, but
ellipsoidality is quite apparent along the boundaries of the corpus
callosum and around the projections into grey matter.  The test
statistic for unimodality in Figure~\ref{figure-slice2}(e) complements
the ellipsoidality results quite well, picking out dominant unimodal
structures (e.g., the voxels dominated by the corpus callosum and to a
lesser degree the cingulum) around which the ellipsoidality measure is
finding more complex voxels.  Finally, the test statistic for
multimodality in Figure~\ref{figure-slice2}(f) clearly identifies
voxels where the three dominant white-matter structures in this ROI
converge, and all other statistics fail to detect any specific
structure.  The statistical summaries developed here provide
complementary information about white-matter microstructure in
clinically acquired data.

%%%%%%%%%%%%%%%%%%%%%%%%%%%%%%%%%%%%%%%%%%%%%%%%%%%%%%%%%%%%%%%%%%%%%%%%%%
\begin{figure}[!htbp]
    \begin{minipage}[]{0.45\textwidth}
      \centering
      \includegraphics*[width=\textwidth]{slice2_slice+rect.eps}
    \end{minipage}
    \begin{minipage}[]{\textwidth}
      \centering      
      \includegraphics*[width=.85\textwidth]{slice2_statistics+points.eps}
    \end{minipage}
  \caption{Axial slice from clinical HARDI acquisition.  Color-coded
    fractional anisotropy (FA) for the whole slice is displayed in
    {\bf (a)} along with the boundaries for the ROI.  For the
    zoomed-in ROI: color-coded FA {\bf (b)}, anisotropy $p$-values
    {\bf (c)}, ellipsoidality $p$-values {\bf (d)}, unimodality test
    statistic {\bf (e)} and multimodality $p$-values {\bf (f)}.
  \label{figure-slice2}}
\end{figure}
%%%%%%%%%%%%%%%%%%%%%%%%%%%%%%%%%%%%%%%%%%%%%%%%%%%%%%%%%%%%%%%%%%%%%%%%%%

We focus on a few specific voxels in Figure~\ref{figure-slice2} using
the Funk Radon Transform without smoothing.  As recommended by Tuch,
we have taken the standardized raw FRT to the power five when plotting
in order to emphasize structure.  Figures~\ref{fig:indvoxel}(a)
and~\ref{fig:indvoxel}(b) show the two most anterior voxels that are
plotted in \ref{figure-slice2} (indicated by yellow dots).  This tract
appears to be ``bending'' as we move from anterior to posterior,
indicated by the shift in direction of the dominant direction seen in
the FRTs.  The statistics quantify this behaviour; the $p$-values for
asymmetry are 0.14 and 0.02 respectively (indicating that the
posterior-most voxel is bending more).  The unimodality of the
anterior voxel is seen from the large unimodality statistic in
Figure~\ref{figure-slice2}(e).  We then look at a voxel in a more
heterogeneous area, where the major fiber tracts appear to merge: the
statistics here indicate multi-modality dominates as is seen in
\ref{fig:indvoxel}(c) and backed up by
Figures~\ref{figure-slice2}(c)-(f).  We observe the most central voxel
has summary statistics that are ellipsoidal but not asymmetric,
clearly observed in Figure~\ref{fig:indvoxel}(d).  The clinical data
have provided evidence at a voxel level, backed up by statistical
hypothesis testing and observed in the FRT visualizations, that
interesting white-matter microstructure may be detected and
characterized using the methodology proposed here.

\section{Discussion}

We have introduced a new set of tools for characterizing orientational
structure from HARDI measurements directly in $q$-space.  This
methodology is unique when compared with existing methods that rely on
reconstructing the spatial information from $q$-space by different
methods of marginalizing the spatial distribution; i.e. from
calculating a spatial ODF.  An ODF has a different meaning if
calculated directly from a Gaussian model, from the non-parametric FRT
(average orientational distribution over all radii without using the
correct volume increment for a marginal PDF) or using PAS-MRI
(orientational distribution associated with a {\em single spatial
  radius} or scale).  In general the magnitude associated with an ODF
is not comparable between methods, neither is the distribution of
noise artifacts.  Our methodology is technically linked to the FRT,
but unlike the FRT we are not constrained to scalar measures
calculated from averages on great circles in $q$-space, and our
methods do not depend on appropriate marginalization to produce
summaries.  The interpretation of our statistical summaries is
straightforward but we note that improvements in data acquisition,
such as increased sampling of directions and the addition of several
shells, will improve estimation of features from the diffusion~PDF
even further \citep{Wisco07}.

Most established methods for characterizing features in white-matter
microstructure have focused on the problem of determining the number
and orientation of peaks in the diffusion~PDF.  None of the
``magnitude'' information of these solutions is comparable or indeed
interpretable apart from DTI based models.  \citet{Savadjiev2006} have
already commented on the unsuitability of such magnitudes as
quantitative measures.  The problem with this fact, and the non-linear
transformation often employed for representing $q$-ball estimates, is
that the coherent treatment of noise artifacts becomes much more
difficult.  The advantage of our theoretical framework developed for
summary statistics is that we may perform hypothesis tests using
critical values that are {\em not} functions of unknown parameters.
We stress that simulation studies for features of diffusion~PDFs are
in general misleading unless the proposed summaries are true
statistics; i.e., their distributions under null hypotheses are
parameter independent.  For example, critical values determined from
Monte Carlo studies for a given diffusion~PDF will not (in general) be
applicable to other diffusion processes than the simulated process
since these critical values are parameter dependent.  This can be
compared to calculating a simple mean rather than a $t$-statistic.  If
we try to elicit the distribution of the sample mean using simulations
at fixed variances, then these critical values are only useful for
variables with the {\em same} variance.

%%%%%%%%%%%%%%%%%%%%%%%%%%%%%%%%%%%%%%%%%%%%%%%%%%%%%%%%%%%%%%%%%%%%%%%%%%
\begin{figure}[!htbp]
    \begin{minipage}[]{0.20\textwidth}
      \centering
      (a)
      \includegraphics*[width=\textwidth]{3257new.ps}
    \end{minipage}
    \begin{minipage}[]{0.20\textwidth}
      \centering
      (b)
      \includegraphics*[width=\textwidth]{3457new.ps}
    \end{minipage}
    \begin{minipage}[]{0.22\textwidth}
      \centering
      (c)
      \includegraphics*[width=\textwidth]{3659.ps}
    \end{minipage}
    \begin{minipage}[]{0.22\textwidth}
      \centering
      (d)
      \includegraphics*[width=\textwidth]{3763.ps}
    \end{minipage}
  \caption{The raw FRT of a set of chosen voxels indicated by yellow
    dots in Figure \ref{figure-slice2}. These are plotted in order of
    decreasing $x_2$-coordinate (or going from the top of the image to
    the bottom).}
  \label{fig:indvoxel}
\end{figure}
%%%%%%%%%%%%%%%%%%%%%%%%%%%%%%%%%%%%%%%%%%%%%%%%%%%%%%%%%%%%%%%%%%%%%%%%%%

Various non-parametric procedures have been proposed to summarize
HARDI data using more than its estimated orientation; e.g., by
investigating the model order of spherical harmonic decomposition
\citep{fra:characterization,ale-bar-arr:detection,Descoteaux:2006}.
\citet{chen2005} modeled the ADC using a product of a truncated
spherical harmonics series.  In general an {\em infinite} order of
spherical harmonic terms should be taken to approximate an arbitrary
Gaussian mixture, but they argued that a crossing fiber should be
sufficiently reproduced by such a truncated representation, and
expressed its complexity using the normalized terms in the spherical
expansion.  Other representations include expressing the ADC in terms
of higher order tensors and spherical harmonics
\citep{Descoteaux:2006}, or just via a spherical harmonic
representation \citep{fra:characterization}.
%, a method similar to \citep{chen2005}.  
Second-order terms in a spherical harmonic decomposition contribute to
describing a single-tensor fiber, but in general more complicated
structure needs to be described in terms of corresponding PDF spatial
properties directly, rather than the fourth- and higher-order terms,
which give too much freedom in structure to be a precise tool for the
description of fine spatial features. Other measures of the entropy of
the PDF have been proposed \citep{rao2004}.

Rather than solely focusing only on the number of peaks in the
diffusion~PDF, we have characterized white-matter microstructure
through the diffusion~PDF directly in $q$-space without parametric
assumptions or imposing smoothness constraints, as we use a
variable bandwidth estimator rather than employing a fixed bandwidth
smoother \citep{OlhedeWhitcher}.  The tissue microstructure is
identified as variation in summary statistics that deviate from a
simple, symmetric model for the diffusion~PDF and is characterized in
behavior relative to the identified dominant great circle in
$q$-space.  Ellipsoidal diffusion~PDFs \eqref{ellipsoid} are simple in
structure and imply the existence of a dominant great circle.  The
deformed ellipsoid class is less stringent in structure, and permits
asymmetric decay in minor axes -- for example, \eqref{deformedell} --
while still conforming to the existence of a dominant great circle.
We describe the precursor to forking by either a deformed ellipse or a
mixture model, to capture further asymmetric structure.  We
differentiate between different white-matter microstructure by
examining variation over that great circle, or variation perpendicular
to the great circle.  Allowing for a greater variety of structure in a
unidirectional diffusion~PDF implies that the power to detect
multi-modal diffusion is necessarily reduced compared to using a
parametric multi-model model, if the proposed parametric model is
correct.  We characterized single peak densities by additional
summaries, such as the anisotropy statistic, the decay ratio statistic
and the asymmetry statistic.  The synthetic forking fiber in
Fig.~\ref{fig:evolution} shows an evolution of such measures as we go
between a single fiber, and a forking fiber.  The synthetic crossing
fiber in Fig.~\ref{fig:evolution} does not exhibit the same
asymmetries.

If one enforces a strict Gaussian (single diffusion tensor) model,
then all variation away from symmetry around the dominant direction
will be interpreted as evidence for a multi-modal diffusion
\citep{par-ale:pico,hos-wil-ans:inference,Behrens2007}.
Modelling using non-Gaussian PDFs allows us to fit asymmetric
structure, rather than just the model indicating a lack of fit of a
single peak. However, using such models leads to a loss of power if a
Gaussian mixture model is appropriate.  Caution should be exercised in
order to protect against over-interpreting fitted models.  With a
model that only includes a family of mixtures of Gaussian diffusion
processes, one is constrained to estimate a Gaussian mixture, however
for a small number of sampled directions there are inevitably issues
with identifiability.  The same realizations may in some cases
equivalently be derived from a unimodal diffusion~PDF with asymmetric
structure or a Gaussian mixture model.  If one chooses to select one
model rather than the other (i.e., choose an asymmetric and scalene
PDF or multiple-tensor), then this decision is based more on the
underlying assumptions of the model rather than on the evidence
directly provided by the observed data.  A large (possibly infinite)
collection of Gaussian diffusion processes may be used to approximate
an observed set of measurements to an arbitrary accuracy, but one has
to consider the possibility that the information being fitted is noise
instead of signal.  We believe the rule of parsimony should be
exercised at all times, and that summaries of orientational structure
can be estimated and interpreted in $q$-space rather than using
(potentially) over-parameterized models.

One potential application of the these $q$-space summaries would be to
improve fiber-tracking algorithms, similar to the use of the Hessian
of a local peak to improve probabilistic tractography models
\citep{Seunarine}.  These summaries would be used in addition to
directions, to allow more careful tracking through forking and fanning
structures ({\em cf} Figure~\ref{figure-slice2}), and distinguish
local structure more consistently with crossing from such features
using both the asymmetry and ellipsoidality measures.

\appendix

\section{Distributions for Test Statistics}
%\label{DistTest}

\subsection{The $T$ Statistic}
\label{disttm}

By performing a Taylor expansion we have 
\begin{equation}
  T \stackrel{d\,|\,H_0}{=} \frac{\tilde\sigma}{\cA} \left[m^{-1/2}(\vare_1-\vare_2) + (\vare_4-\vare_3)\right] + O\left(\tilde\sigma^2\right),
\end{equation}
where $\vare_1$ and $\vare_3$ ($\vare_2$ and $\vare_4$) are
distributed as the maxima (the minima) of $m$ Gaussian random
variables.  Assume for a given acquisition scheme that the great
circle passes near $m$ independent observations.  We find that
\begin{equation}
  \max_k\{\wh\cA\left(\hat{\q}_k\right)\} \approx \cA + \tilde\sigma
\vare_3 \quad \text{and} \quad  \min_k\{\wh\cA\left(\hat{\q}_{k}\right)\} \approx \cA + \tilde\sigma\vare_4,
\end{equation}
where $\vare_3$ and $\vare_4$ are random variables.  Then the
distribution of $\vare_3$ may be approximated as the maximum of $m$
Gaussian random variables, and likewise with $\vare_4$ as the minimum.
As we linearly interpolate the value the extremum will, in the worst
case, behave like the extremum of the $m$ variables.  Thus
\begin{equation}\label{e:Taylor2}
  \max_j\{{\wh\cA}\left(\alpha_{j}\right)\} = \cA +
  \frac{\tilde\sigma}{\sqrt{m}} \, \vare_1 \quad \text{and} \quad
  \min_j\{{\wh\cA}\left(\alpha_{j}\right)\} = \cA +
  \frac{\tilde\sigma}{\sqrt{m}} \, \vare_2,
\end{equation}
where $\vare_1$ and $\vare_3$ ($\vare_2$ and $\vare_4$) are maximum
(minimum) order statistics over $m$ Gaussian random variables.  Note
that maximum and minimum order statistics \citep{Walsh1970} can be
approximated as independent even for small values of $n$, and that
extrema in the perpendicular great circles will be under $H_0$
independent of extrema over the dominant great circle.

Assume that $0<\tilde\sigma<\sigma$ is the variance at the observation
points after constructing the multiresolution-based estimator
$\wh\cA(\tilde\q_j)$ and an estimator of $\sigma$ is provided by the
multiresolution algorithm, and will be denoted $\wh{\sigma}^{\ast}$
\citep{OlhedeWhitcher}.  We must now estimate $\cA$ and
$\tilde\sigma$, as indicated in \eqref{e:estunderh0}.  We have
approximated the interpolated observations as Gaussian random
variables, where their variance will vary from observation to
observation due to the estimation scheme and the linear interpolation.
To be able to determine a suitable threshold we must determine the
distribution of the test statistic $U$
\begin{equation}
  P\left(U\le{u}\right) \overset{d|H_0}{=} P\left(m^{-1/2}
  (\vare_1-\vare_2) + (\vare_4-\vare_3) \le u\right) +
  O\left(\frac{\tilde\sigma}{\cA}\right).
\end{equation}
Thus, the PDF of $U$ is approximately given by
\begin{eqnarray}\label{distributiondir}
  f_1(e) &=& f_3(e) = f_2(-e) = f_4(-e) = m [\Phi(e)]^{m-1} \phi(e)
  \nonumber\\
  f_U(u) &=& m f_1\left(e\sqrt{m}\right) \ast f_1\left(e
  \sqrt{m}\right) \ast f_1(e) \ast f_1(e)\\ 
  F_U(u) &=& \int_{-\infty}^{u} f_U(v) \, dv. \nonumber
\end{eqnarray}

Under the alternative hypothesis that the ``the density is unimodal
and prolate,'' the expected diffusion on the dominant great circle
taking the value $\cA$ and on the perpendicular point taking the value
$\cA_\text{min}$, we have with $\tilde{\sigma}$ the smoothed variance
and $\breve{\sigma}$ the interpolated variance
\begin{eqnarray}
  T &=& \left[\frac{\max_j\{\wh{\cA}_\perp\left(\alpha_j \right)\}}
    {\min_j\{\wh{\cA}_\perp\left(\alpha_{j}\right)\}}\right]
  \left[\frac{\max_k\{\wh{\cA}\left(\hat\q_{k}\right)\}}
    {\min_k\{\wh{\cA}\left(\hat\q_{k}\right)\}}\right]^{-1} - 1 \nonumber\\
  &\overset{d\,|\,H_1}{=}&
  \left[\frac{\cA+\breve{\sigma}\eta_1}{\cA_\text{min}+\breve{\sigma}\eta_2}\right]
  \left[\frac{\cA+\tilde\sigma\eta_3}{\cA+\tilde\sigma\eta_4}\right]^{-1} - 1
  \nonumber\\
  &=& \frac{\cA}{\cA_\text{min}} - 1 + \frac{1}{\cA_\text{min}}
  \left[\breve{\sigma}\eta_1-\tilde{\sigma}\eta_3+\tilde{\sigma}
  \eta_4 - \frac{\breve{\sigma}\cA}{\cA_\text{min}}\eta_2\right] +
  \cdots.
\end{eqnarray}
$\eta_3$ and $\eta_4$ are here maximum and minimum order statistics,
respectively, and $\eta_1$ and $\eta_2$ are Gaussian random variables.
We have good estimators for $\breve\sigma$, $\tilde\sigma$ and $\cA$
distinct from those used in $T$, but we are lacking a good independent
estimator for $\cA/\cA_\text{min}$.  This means we cannot start from a
null hypothesis of prolate and unimodal.  Let us consider the same
object under multi-modality, with $\cA_{\perp\max}$ and
$\cA_{\perp\min}$ the maximum or minimum of the the perpendicular
great circle:
\begin{eqnarray}
  T &\overset{d\,|\,H_0'}{=}&
  \left[\frac{\cA_{\perp \max}+\breve{\sigma}\eta_5}{\cA_{\perp \min}+\breve{\sigma}\eta_6}\right]
  \left[\frac{\cA_\text{max}+\breve\sigma\eta_7}{\cA_\text{min}+\breve\sigma\eta_8}\right]^{-1} - 1\\
  &=& \breve{\sigma} \left(\frac{\eta_8}{\cA_{\min}} +
    \frac{\eta_5}{\cA_{\perp\max}} - \frac{\eta_6}{\cA_{\perp\min}}
    - \frac{\eta_7}{\cA_\text{max}}\right) \nonumber
  \cdots,
\end{eqnarray}
where $\{\eta_i\}$ is a set of Gaussian variables.  The distribution
of $T$, under the assumption that
\begin{equation}
  \frac{\cA_{\perp\max}\cA_\text{min}}{\cA_{\perp\min}\cA_\text{max}}
  \approx 1,
\end{equation}
which we shall refer to as $H_0'$, is therefore given approximately by
the above expression.  The test statistic $\wt{U}$ is calculated under
a $H_0'$ of multimodality, thus
$(\wt{\cA}_{\max}\cA_\text{min})/(\wt{\cA}_{\min}\cA_\text{max})\le c$
for some specified constant such as $c=2$. Then
\begin{eqnarray}
  \tilde{T}&\overset{d\,|\,H_0'}{=}&[c-1] +
  \frac{\breve{\sigma}}{\cA_{\min}}\left(c\wt{\eta}_8 +
  \frac{c\cA_{\min}}{\wt{\cA}_{ \max}}\wt{\eta}_5-\frac{c
    \cA_{\min}}{\wt{\cA}_{ \min}} \wt{\eta}_6- \frac{c
    \cA_{\min}}{\cA_\text{max}}\wt{\eta}_7\right)+\cdots,
\end{eqnarray}
where $\{\wt{\eta}_l\}_{l=5}^8$ will be Gaussian random variables.
The alternative hypothesis will have $(\wt{\cA}_{\max}\cA_\text{min})
/ (\wt{\cA}_{\min}\cA_\text{max}) > c$, and so this is a one-sided
test.  We assume that $\cA_{\min}\approx\wt{\cA}_{\min}$ and that
$c\cA_{\min}/\wt{\cA}_{\max}\approx{c\cA_{\min}/\cA_\text{max}}
\approx 1$ (these could be replaced by inserting estimators, and most
frequently correspond to conservative choices) to produce
\begin{equation}
  \wt{T}-[c-1] \overset{d\,|\,H_0'}{=} N\left(0,
  \frac{\breve{\sigma}^2}{\cA_{\min}^2} \left(2c^2 + 2\right)\right) +
  \cdots.
\end{equation}
We estimate $\cA_{\min}$ by
$\wh{\cA}_{\min}=\wh{\cA}(\wh{\bm{\upsilon}}_1)$ to get $\tilde{U}$,
and replace $\breve{\sigma}$ by the (conservative) estimate of
$\wh{\sigma}_{\cA}$.

\subsection{The $X$ Statistic}
\label{threshxi}

Under a null hypothesis of isotropy we find that 
\begin{equation}
  X-1 \overset{d|H_0}{=} \frac{\tilde{\sigma}\eta_9 -
    \breve{\sigma}\eta_{10}}{\cA\log\cA} +
  O\left(\tilde{\sigma}^2\right),
\end{equation}
where $\eta_9$ is the maximum order statistic and $\eta_{10}$ may be
approximated as a Gaussian random variable.  We can approximate the
density of $Q$ via
\begin{equation}
  f_{Q|H_0}(q|H_0) = \rho^{-1/2}\phi\left(\frac{q}{\sqrt{\rho}}\right)
  \ast \left[m \Phi^{m-1}\left(-\frac{q}{\rho}\right) \rho^{-1}
    \phi\left(-\frac{q}{\rho}\right)\right],
\end{equation}
for $-\infty<q<\infty$, where $\ast$ denotes the convolution operator,
$\Phi(\cdot)$ and $\phi(\cdot)$ are the standard Gaussian CDF and PDF,
respectively.

\subsection{The $Z$ Statistic}
\label{threshzeta}

To derive the $Z$ statistic we note that
with $\varepsilon_5$ an aggregation of a maximum over $m$ Gaussian
random variables, and an independent Gaussian
observation, and $\varepsilon_6$ is a Gaussian random variable
\begin{eqnarray}
  Z &=& \frac{\log\cA\left(\gamma\right) +
    (\tilde\sigma\vare_5)^{-1}\cA(\gamma)}{\log\cA(\iota) +
    (\breve\sigma\vare_6)^{-1}\cA(\iota)}\\
  &=& \frac{\log\cA(\gamma)}{\log\cA(\iota)} +
  \frac{\tilde\sigma\vare_{5}}{\cA(\gamma)\log\cA(\iota)} -
  \frac{\breve\sigma\vare_{6}\log\cA(\gamma)}{\cA(\iota)\log^2\cA(\iota)},
\end{eqnarray}
where $\gamma=\q_{k_\text{max}+M}$, $\iota=\q_{k_\text{max}+M+N/4}$
and $M=N/(2m)$.  Under the null $\cA(\gamma)=\cA(\iota)$ and so
\begin{eqnarray}
  Z &\overset{d\,|\,H_0}{=}& 1 +
    \frac{\tilde{\sigma}\vare_{5}}{\cA(\iota)\log\cA(\iota)} -
    \frac{\breve{\sigma}\vare_{6}}{\cA(\iota)\log\cA(\iota)}\\
  V &=& \frac{(Z-1)\overline{\cA}_N\log\overline{\cA}_N}{\hat\sigma_2} 
    = \vare_5 - \rho^{-1/2}\vare_6 = \vare_7,
\end{eqnarray}
where $\vare_{5}$ and $\vare_6$ are  due to
properties of the interpolation.  The random variables $\vare_5$ and
$\vare_6$ may be approximated as independent.  Note that the PDF of
$\vare_7$ is approximated by the following density
\begin{equation}
  F_{\vare_7}(e) = P(\epsilon_7\le e), \quad
  g(e) = \frac{25m}{2\sqrt{2}} \left[\Phi^{m-1}\left(5e'\right)
  \phi\left(5e'\right) \ast
  \phi\left(\frac{5}{2\sqrt{2}}e'\right)\right](e),
\end{equation}
and $f_{\vare_7}(e)=(g{\ast}g)(e)$.  Depending on the value of $m'$ we
find that the approximation performs well.  
% The reason why we do not use the interpolated variance for the
% convolved Gaussian is that the sum is equivalent to averaging over
% three {\em independent} observations with an appropriate variance.

\subsection{The $K$ Statistic}
\label{threshkappa}

Under the null hypothesis $K$ will be approximately distributed as the
absolute value of a Gaussian random variable with mean zero and
variance given by $\var\{K\}\approx4(1.35)^2m^{-1}\var\{P_k\}$, where
the factor multiplying the variance of $P_k$ is chosen to reflect the
degrees of freedom of the sampled sphere.  We note that the variance
of $P_k$ is given by approximated by
\begin{equation}
  \var\left\{P_k\right\} \approx \frac{32\tilde\sigma^2}
    {m\left[\sum_{j=1}^N{\cA}_{\perp}\left(\alpha_j\right)/N\right]^2}
    \Rightarrow
    \var\left\{K\right\}=\frac{2^7\tilde\sigma^2(1.35)^2}
    {m^2\left[\sum_{j=1}^N{\cA}_{\perp}\left(\alpha_j\right)/N\right]^2}.
\end{equation}
The normalized statistic is thus defined by
\begin{equation}
  R = \frac{mK\left[\sum_{j=1}^N\wh{\cA}_{\perp}\left(\alpha_j\right)/N\right]}{2^{7/2}(1.35)\hat\sigma_2} = \frac{R_2}{2^{3/2}(1.35)} \sim 2\phi(r),\quad r>0.
\end{equation}

\bibliographystyle{imsart-nameyear}
\bibliography{brandon}

\end{document}
